\documentclass[oneside, 10pt]{amsart}
\usepackage{amscd, amsmath, amssymb, amsthm, amsfonts, amstext, geometry, verbatim, enumitem, graphicx, mathtools, xfrac, microtype, nameref, thmtools}
\usepackage[breaklinks=true]{hyperref}
\usepackage[capitalize]{cleveref}
\usepackage[hyperref=true, backend=bibtex, firstinits=true, citestyle=numeric-comp, sortlocale=en_US, url=false, doi=false, eprint=true, maxbibnames=4]{biblatex}            
%\usepackage[notref,notcite]{showkeys}
\usepackage[matrix,arrow,curve]{xy}
\usepackage[toc,page]{appendix}

\addbibresource{question.bib}
\renewbibmacro*{volume+number+eid}{\ifentrytype{article}{\- \iffieldundef{volume}{}{Vol.~\printfield{volume},}\iffieldundef{number}{}{ No.~\printfield{number},}}}
\renewbibmacro{in:}{\ifentrytype{article}{}{\printtext{\bibstring{in}\intitlepunct}}}
\newbibmacro{string+doi}[1]{\iffieldundef{doi}{\iffieldundef{url}{#1}{\href{\thefield{url}}{#1}}}{\href{http://dx.doi.org/\thefield{doi}}{#1}}}
\DeclareFieldFormat[article, inproceedings, inbook, book, thesis]{title}{\usebibmacro{string+doi}{\mkbibquote{#1}}}
\renewcommand*{\bibfont}{\footnotesize}

\newlist{lemlist}{enumerate}{1} \setlist[lemlist]{label={\rm(\arabic{lemlisti})}, ref=\thelemma.(\arabic{lemlisti}),noitemsep} \Crefname{lemlisti}{Lemma}{Lemmas}

\theoremstyle{plain}

\newtheorem{thm}{Theorem}
\Crefname{thm}{Theorem}{Theorems}
\numberwithin{equation}{section}

\newtheorem{lemma}{Lemma}
\numberwithin{lemma}{section}
\Crefname{lemma}{Lemma}{Lemmas}

\newtheorem{cor}[lemma]{Corollary}
\Crefname{cor}{Corollary}{Corollaries}

\newtheorem{prop}[lemma]{Proposition}
\Crefname{prop}{Proposition}{Propositions}

\newtheorem*{thm*}{Theorem}
\newtheorem*{lemma*}{Lemma}

\theoremstyle{remark}

\newtheorem{rem}[lemma]{Remark}
\Crefname{rem}{Remark}{Remarks}

\theoremstyle{definition}

\newtheorem{df}[lemma]{Definition} \Crefname{df}{Definition}{Definitions}
\newtheorem{example}[lemma]{Example} \Crefname{example}{Example}{Examples}

\newcommand{\catname}[1]{{\normalfont\textbf{#1}}}
\DeclareMathOperator{\HH}{H}
\DeclareMathOperator{\colim}{colim}
\DeclareMathOperator{\Ker}{Ker}
\DeclareMathOperator{\Hom}{Hom}
\DeclareMathOperator{\sd}{sd}
\DeclareMathOperator{\hh}{\overline{{\it h}}}
\newcommand{\ZZ}{\mathbb{Z}}
\newcommand{\pushoutcorner}[1][dr]{\save*!/#1+1.2pc/#1:(1,-1)@^{|-}\restore}
\newcommand{\pullbackcorner}[1][dr]{\save*!/#1-1.2pc/#1:(-1,1)@^{|-}\restore}

\title[Parametrized symmetric groups]{Parametrized symmetric groups and \\ nonabelian tensor products}
\keywords {Extensions of type $\mathfrak{H}_n(G)$, amalgamated products, van Kampen theorem,
 {\em Mathematical Subject Classification (2010):} 55Q05, 20B30, 20E06, 20J06, 20F05, 20F55}

\author {Sergey Sinchuk}
\address{Chebyshev laboratory, St. Petersburg State University, St. Petersburg, Russia}
\email {sinchukss at gmail.com}
\date {\today}

\begin{document}

\begin{abstract} By amalgamating copies of symmetric group $S_n$ we construct a central extension of the wreath product $G \wr S_n$.
We then relate this group with extensions of type $\mathfrak{H}_n(G)$ introduced by U.~Rehmann in 1970's. 
We also describe a topological application of this construction.
 %the fundamental group of the fiber of the ``alternating`` map $BG^{2n}\to \mathbb{Z}[BG]$.
\end{abstract}

\maketitle

\section{Introduction}
The aim of this note is to describe a rather surprising connection between symmetric groups and nonabelian tensors products 
 which apparently has not been studied in detail previously.
 
\begin{comment}
Denote by $D_n(G)$ the subgroup of $G^n$ consisting of vectors
 $(g_1,\ldots, g_n)$ for which the product $g_1\cdot \ldots \cdot g_n$ lies in the derived subgroup $[G, G]$.
Recall from Rehmann's paper~\cite{Reh78} that the universal {\it extension of type $\mathfrak{H}_n$} (denoted $H_n(G)$) 
 is, by definition, a certain (central) extension of $D_n(G)$ explicitly presented by generators and relations 
 (see section~\ref{sec:Hnextensions} below).
The extension $H_n(G)$ and its ''antisymmetrization`` $H^\wedge_n(G)$ can be characterised by the property that 
 they fit into the following pullback squares:
\[ \xymatrix{ G \mathbin{\widetilde{\wedge}} G \pullbackcorner \ar@{^{(}->}[d] \ar@{->>}[r] & [G, G] \ar@{^{(}->}[d]^{\iota_1} \\ H_n(G) \ar@{->>}[r] & D_n(G)} \qquad
   \xymatrix{ G \wedge G \pullbackcorner \ar@{^{(}->}[d] \ar@{->>}[r] & [G, G] \ar@{^{(}->}[d]^{\iota_1} \\ H^\wedge_n(G) \ar@{->>}[r] & D_n(G)} \]
Here $G \wedge G$ denotes the nonabelian exterior square of $G$ and  
 $G \mathbin{\widetilde{\wedge}} G$ is the modern notation for the Dennis group $(G, G)$ defined in~\cite{De76} (also denoted $U_G$ by Rehmann in \cite{Reh78}).
The map $\iota_1$ in the above diagrams is defined by $\iota_1(g) = (g, 1,\ldots, 1)$.
\end{comment}
Let $G$ be a (possibly noncommutative) group. 
Denote by $S_n(G)$ the group generated by elements $s_i(a)$, where $i$ is an integer $1\leq i\leq n-1$ and $a$ is an element of $G$, subject to the relations
\begin{align}
s_i(a)^2                 = &\, 1,                                     & a \in G,                  \label{Cox1} \\
%s_{i+1}(b) s_i(a) s_{i+1}(b) = &\, s_i(c) s_{i+1}(c^{-1}ab) s_i(c),& a, b, c \in G, \label{S2} \tag{SnG2} \\
s_i(a) s_{i+1}(b) s_i(c) = &\, s_{i+1}(a^{-1}cb) s_i(a) s_{i+1}(b),   & a, b, c \in G,            \label{Cox2}  \\
s_i(a) s_j(b)            = &\, s_j(b) s_i(a).                         & |i-j|\geq 2,\, a, b\in G. \label{Cox3} 
\end{align}
Notice that in the special case $G=1$ the above presentation boils down to the usual Coxeter presentation of the symmetric group $S_n$,
 i.\,e. $S_n(1) = S_n$.
 
\begin{comment}
Denote by $F$ the free product of copies of the symmetric group $S_n$ in which each copy is indexed by an $n$-tuple $(g_1, \ldots, g_n) \in G^n$.
Denote by $S_n(G)$ the quotient of $F$ modulo relations of the form $s_{(g)}$ = $s_{(h)}$ in which $g, h \in G^n$ and $s\in S_n$ are such that in the wreath product $G \wr S_n = G^n \rtimes S_n$ the equality $s^g = s^h$ holds
 (here $s_{g}$ and $s_{h}$ denote images of $s \in S_n$ under respective embeddings $S_n\to F$).
It is not hard to check that there is a well defined map $\mu_n \colon S_n(G)\to G \wr S_n$ given by $s_{(g)} \mapsto s^g$.
\end{comment}
The group defined above bears some similarity with the {\it parametrized braid group} defined by Loday---Stein in~\cite{LS05} 
 (cf. also with Kassel---Reutenauer, \cite{KR98}). By analogy, we call the group $S_n(G)$ a {\it parametrized symmetric group}.
Our first main result is the following theorem, which gives a concrete description of the group $S_n(G)$.
\begin{thm} \label{thm:summary} For $n \geq 3$ there is a map $\mu_n\colon S_n(G) \to G \wr S_n$ which is a crossed module.
The cokernel and the kernel of $\mu_n$ are isomorphic to $\HH_1(G, \ZZ)$ and $\HH_2(G, \ZZ)$ respectively.
Moreover, the group $S_n(G)$ is isomorphic to $S_n \ltimes H_n^\wedge(G)$. \end{thm}
Here $G\wr S_n$ stands for the wreath product of $G$ and $S_n$ and $H_n^\wedge(G)$ denotes the quotient of the so-called 
 ''universal extension of type $\mathfrak{H}_n(G)$`` (which is denoted $H_n(G)$) 
modulo all ''symbols`` of the form $\{x, x\} = 1$, $x\in G$.

Extensions of type $\mathfrak{H_n}$ defined by U.~Rehmann in~\cite{Reh78} played a key role in his proof of generalized Matsumoto theorem.
The groups $H_n(G)$, $H_n^\wedge(G)$ are directly related with nonabelian tensor products and their quotients.
% (which appear under the name of ''extensions of type $\mathfrak{U}$'' in~\cite{Reh78})
More formally,  $G \mathbin{\widetilde{\otimes}} G$ (resp. $G \wedge G$) are precisely the subgroups of 
 $H_n(G)$ (resp. $H_n^\wedge(G)$) generated by symbols $\{x, y\}$ (we refer the reader to Section~\ref{sec:Hnextensions} for more details).
 
Theorem 1 and its proof (which relies solely on basic combinatorial group theory) may be of interest for the following reasons:
\begin{itemize}
 \item It establishes a connection between two classical objects: symmetric groups and the second homology group.
 \item It helps to better understand the nature of extensions $H_n(G)$ appearing in the proof of noncommutative Matsumoto theorem in~\cite{Reh78}.
       For example we obtain a presentation for $H_n(G)$ that is simpler than its original presentation given in~\cite{Reh78}
       (4 instead of 5 relations that are also shorter), see~\cref{prop:simpler}.
       %Another advantage of this presentation is that it is explicitly obtained from the ''transposition-presentation`` of the symmetric group $S_n$ via a rewriting process.
 \item Theorem 1 is motivated by (and provides partial answer to) some purely topological question, which we describe below in more detail.
\end{itemize}

For a pointed simplicial set $(X, x_0)$ denote by $\ZZ[X]$ the associated free simplicial abelian group.
Define the ''alternating`` map of simplicial sets $h_n\colon X^{2n} \to \ZZ[X]$ by the following identity:
 \[h_n(x_1, x_2, \ldots x_{2n}) = \sum\limits_{i=1}^{2n}(-1)^ix_i.\]
Define the ''stabilization`` map $X^{2n} \hookrightarrow X^{2(n+1)}$ by $x \mapsto (x, x_0, x_0)$ and denote by $X^\infty$ the colimit of $X^n$.

Recently S.~Podkorytov showed that the limit map $h_\infty = \mathrm{colim}_n(h_n) \colon X^\infty \to \ZZ[X]$ is a {\it quasifibration}
 provided $X$ is fibrant and connected.
By definition, this means that the natural map $h^{-1}_\infty(b) \to F_{h_\infty}(b)$ between the fiber and the homotopy fiber of $h_\infty$ over any point $b$ is 
a weak (hence homotopy) equivalence.
This assertion is essentially proved in~\cite[Lemma~9.1]{Po17}, however some technical work is required to
  formulate it this way, see~\cref{lm:weak-equiv} below.

One may wonder whether the maps $h_n$ satisfy some weaker analogue of this property,
 i.\,e. one might expect that the map $h_n^{-1}(0) \to F_{h_n}(0)$ is $c(n)$-connected for some $c(n)$ which depends only on $n$ and tends to infinity as $n\to \infty$.
  
The following result which is a consequence of Theorem 1 asserts that this map is at least $1$-connected in one important special case.
\begin{thm} \label{thm:main} If $X=BG$ is the classifying space of a group $G$ then for $n\geq 3$
 the natural map $h_n^{-1}(0) \to F_{h_n}(0)$ induces an isomorphism of fundamental groups
 $\pi_1(h^{-1}_n(0))\cong \pi_1(F_{h_n})$. \end{thm}
\cref{thm:main} reduces to~\cref{thm:summary} by means of Quillen Theorem A and van Kampen theorem.
It also involves some explicit computations with simplicial fundamental groups.

\subsection{Acknowledgements}
I wish to thank Semyon Podkorytov for suggesting the problem and also for his kind permission to use some of his unpublished results.
I am also grateful to S.~O.~Ivanov and V.~Isaev for their numerous helpful comments and interest in this work.
This work received financial support from the Russian Science Foundation grant 14-21-00035.

\section{Definition of $S_n(G)$ and its basic properties} 
%\subsection{The definition of $S_n(G)$} \label{sec:QnG-def}
The aim of this section is to prove the first part of~\cref{thm:summary}, namely the fact that $S_n(G)$ is a crossed module over $G \wr S_n$.
It will be more convenient for us to start with another definition of the group $S_n(G)$ given in terms of amalgams of symmetric groups.
In the end of the section we show that this alternative definition coincides with the one given in the introduction.

Let $G$ be a group. Recall that the {\it wreath product} is, by definition,
the semidirect product $G^n \rtimes S_n$, in which $S_n$ acts on $G^n$ on the right by permuting its factors.

Consider the family $\{{S_n}^{(g)}\}_{g\in G^n}$ of isomorphic copies of $S_n$ indexed by elements of $G^n$ and let $F$ be the free product of groups from this family.
For $s\in S_n$ and $g\in G^n$ we denote by 
$s_{g}$ the image of $s$ in $F$ under the canonical map $S_n^{(g)} \to F$.
%Recall that we defined the group $S_n(G)$ as the quotient of the free product $F$ of copies of $S_n$ indexed by $G^n$ modulo the following single family of relations:
\begin{equation} \label{eq:main_rel} s_{g} = s_{h}, \text{ if $s^g = s^h$ holds in $G \wr S_n$.} \end{equation}
We denote by $s_g$ the image of the element $s_g\in F$ under the canonical map $F\to S_n(G)$.

\begin{rem}
Expanding the definition of the semidirect product we get that
\begin{equation} \nonumber s^g = (1_{G^n}, s)^{(g, 1)} = (g^{-1}, s) (g, 1) = (g^{-1} g^{(s^{-1})}, s). \end{equation}
therefore the equality $s^g = s^h$ holds in $G \wr S_n$ iff
$g^{-1} g^{(s^{-1})} = h^{-1} h^{(s^{-1})}$, or what is the same, iff $s$ fixes $hg^{-1}$.

The last statement immediately implies that the map $\mu \colon S_n(G) \to G \wr S_n$ given by $\mu(s_g) = s^g = (g^{-1}g^{(s^{-1})}, s)$
is well-defined. %We will see shortly that $\mu$ is, in fact, a {\it crossed module} in the sense of~\cite[\S~2.2]{BHS11}.
\end{rem}

Observe from the definition of $S_n(G)$ that there is a split exact sequence.
\begin{equation} \label{eq:ex-seq} \xymatrix{1 \ar[r] & \mathrm{Ker}(\pi) \ar[r] &  S_n(G) \ar@<0.5ex>[r]^{\pi} & S_n \ar[r] \ar@{-->}@<0.5ex>[l]^{\iota(1)} & 1,} \end{equation}
Here the map $\pi = \pi_{S_n} \circ \mu$ removes subscript $g$ from each $s_g$ and the section $\iota(1)$ maps $s$ to $s_1$, where $1$ is the identity element of $G^n$.
Clearly, $S_n(G)$ decomposes as $HS_n(G) \rtimes S_n$, where $HS_n(G) = \mathrm{Ker}(\pi)$.

\subsection{Crossed module structure on $S_n(G)$}
Recall from~\cite[\S~2.2]{BHS11} that a {\it crossed module} is a morphism of groups $\mu\colon M\to N$ together with a right action of $N$ on $M$ 
compatible with the conjugation action of $N$ on itself, i.\,e.
\begin{equation} \label{eq:precrossed} \tag{CM1} \mu(m^n) = \mu(m)^n \text{ for all $n \in N$, $m \in M$}, \end{equation}
which also satisfies the following identity called {\it Peiffer identity}:
\begin{equation} \label{eq:Peiffer} \tag{CM2} m^{m'} = m^{\mu(m')} \text{ for all $m, m' \in M$}.\end{equation}
In our situation, we let $G \wr S_n$ act on $S_n(G)$ by 
\begin{equation} \label{eq:action} (s_g)^{(h, t)} = {s^t}_{(gh)^t}, \text{ for $s, t \in S_n$ and $g, h \in G^n$.} \end{equation}
The goal of this subsection is to prove the following.
\begin{prop} \label{thm:cms} For $n \geq 3$ the map $\mu \colon S_n(G) \to G \wr S_n $ is a crossed module. \end{prop}
From the fact that $\mu$ is a crossed module one can deduce that $\Ker(\mu)$ is a central subgroup of $S_n(G)$ and $\mathrm{Im}(\mu)$ is a normal subgroup of $G \wr S_n$.
It is not hard to show that the group $S_2(G)$ is isomorphic to a free product of copies of $S_2$ (whose center is always trivial).
Therefore, the requirement $n\geq 3$ in the statement of~\cref{thm:cms} is essential.

Verification of the fact that formula~\eqref{eq:action} gives a well-defined action of $G\wr S_n$ on $S_n(G)$ 
 that satisfies~\eqref{eq:precrossed} is lengthy but straightforward. Let us show that~\eqref{eq:Peiffer} holds. 
It suffices to verify Peiffer identities only for the generators of $S_n(G)$, for which it takes the form:
\begin{equation} \label{eq:Peiffer-gen} t^{-1}_h s_g t_h = {s^t}_{(gh^{-1})^t\cdot h} \text{ for all $s, t \in S_n$, $g, h\in G^n$.}\end{equation}
If we act on both sides of the above formula by $(h^{-1}, 1) \in G \wr S_n$ we obtain the equality
$t^{-1}_1 s_{gh^{-1}} t_1 = {s^t}_{(gh^{-1})^t}$.
Thus, to prove~\eqref{eq:Peiffer-gen} it suffices to show the following simpler relation:
\begin{equation} \label{eq:Peiffer-simple} t^{-1}_1 s_g t_1 = {s^t}_{g^t} \text{ for all $s, t \in S_n$, $g\in G^n$.}\end{equation}

The key step in the proof is the following lemma.
\begin{lemma} \label{lem:transp-deff} 
 The relation~\eqref{eq:Peiffer-simple} holds in the special case when $s=(ij)$ and $t=(kl)$ are two nonequal transpositions. \end{lemma}
\begin{proof} 
First of all, we immediately check that~\eqref{eq:Peiffer-simple} holds in the special case when $t$ fixes $g'\in G^n$.
Indeed, by~\eqref{eq:main_rel} we have $t_1 = t_{g'}$, hence 
\begin{equation} \nonumber t^{-1}_1 s_{g'} t_1 = t^{-1}_{g'} s_{g'} t_{g'} = (s^t)_{g'} = (s^t)_{g'^t}.  \end{equation}

Without loss of generality we may assume that $l\neq i$ and $l\neq j$.
Denote by $g'$ the vector which differs from $g$ only at $l$-th position, for which we set $g'_l = g_k$. 
Since the only nontrivial component of $g'g^{-1}$ (resp. $g'g^{-t}$) is located at $l$-th (resp. $k$-th) position,
it is fixed by $s$ (resp. $s^t)$, hence from~\eqref{eq:main_rel} we conclude that $s_g = s_{g'}$ (resp. $(s^t)_{g'} = (s^t)_{g^t}$).
Finally, since $g'$ is fixed by $t$, we get that
\begin{equation} \nonumber t^{-1}_1 s_g t_1 = t^{-1}_1 s_{g'} t_1 = (s^t)_{g'^t} = s^t_{g'} = (s^t)_{g^t}. \text{\qedhere}\end{equation} \end{proof}

\begin{proof}[Proof of~\cref{thm:cms}]
Let us show that~\eqref{eq:Peiffer-simple} holds for arbitrary transpositions $s, t \in S_n$.
It suffices to consider the case $s=t=(ij)$. 
After choosing some $k\neq i,j$ and presenting $(ij)$ as $(kj)(ik)(kj)$ we use \cref{lem:transp-deff}:
\begin{equation} \nonumber (ij)^{-1}_1 (ij)_g (ij)_1 = (ij)^{-1}_1 (kj)_g (ik)_g (kj)_g (ij)_1 =
(ki)_{g^{(ij)}} (jk)_{g^{(ij)}} (ki)_{g^{(ij)}} = (ij)_{g^{(ij)}}. \end{equation}
The proposition now follows by induction on the length of permutations $s$, $t$.
\end{proof}

\subsection{An explicit presentation of $S_n(G)$}
In this section we obtain an explicit presentation of $S_n(G)$ by generators and relations.
This presentation turns out to be much more economical than the one given in the previous subsection.
From this presentation we later obtain an explicit presentation for the subgroup $HS_n(G)$ via Reidemeister-Schreier method.

We start with the following simple lemma which is, in essence, a variant of the standard presentation of the symmetric group in terms of reflections
 with some redundant generators and relations added (cf.~\cite[Theorem~2.4.3]{Ca89}).
\begin{lemma} \label{lm:Snpres} For $n\geq 3$ The symmetric group $S_n$ admits presentation with 
$\{(ij) \mid i\neq j,\ 1\leq i,j\leq n\}$ as the set of generators and the following list of relations
 (in every formula distinct letters denote distinct indices):
\begin{align}
(ij)^2 = &\, 1,         \label{Sym1} \\
(ij)^{(jk)} = &\, (ik), \label{Sym2} \\
[(ij), (kl)] = &\,1.    \label{Sym3} \\
(ij) =&\, (ji),         \label{Sym0}
\end{align}
\end{lemma}
\begin{rem}
To simplify notation, up to the end of this section we adhere to the convention that in our formulae distinct letters denote distinct indices.
\end{rem}

Now we are ready to formulate the main result of this subsection.
\begin{prop} \label{prop:Q-pres} For $n\geq 3$ and arbitrary group $G$ the group $S_n(G)$ admits presentation with 
$\{(ij)_a \mid i\neq j, 1\leq i,j,\leq n, a\in G\}$ 
as the set of generators and the following list of relations. % (as before, distinct letters denote distinct indices):
\begin{align}
(ij)_a^2 = &\, 1,                \label{Q1} \tag{SG1} \\
(ij)_a^{(jk)_b} = &\, (ik)_{ab}, \label{Q2} \tag{SG2} \\
[(ij)_a, (kl)_b] = &\,1.         \label{Q3} \tag{SG3} \\
(ij)_a =&\, (ji)_{a^{-1}}        \label{Q4} \tag{SG4} 
\end{align}
\end{prop}
\begin{proof}
Denote by $S'$ the group from the statement of the proposition.
For $1\leq i\leq n$ and $x\in G$ denote by $x[i]$ the element of $G^n$ 
 whose only nontrivial component equals $x$ and is located in the $i$-th position.

It is not hard to deduce from the definition of $S_n(G)$ and~\cref{lem:transp-deff} 
 that the formula $(ij)_a \mapsto (ij)_{a[j]}$ gives a well-defined map $\varphi\colon S'\to S_n(G)$.

Now we are going to construct the map $\psi\colon S_n(G)\to S'$ in the opposite direction. 
Using the presentation of $S_n$ given by~\cref{lm:Snpres}  we define for a fixed $g\in G^n$ the map 
 $\psi_g\colon S_n \to S'$ by $\psi_g((ij)) = (ij)_{g_i^{-1} g_j}.$
It is obvious that $\psi_g$ preserves the defining relations \eqref{Sym1}--\eqref{Sym0} of $S_n$. 
It remains to show that the equality $\psi_g(s) = \psi_h(s)$ holds whenever $g, h\in G^n$ and $s\in S_n$ are such that $s_g=s_h$.

Indeed, if $hg^{-1}$ is fixed by $s$ then for every $1\leq i\leq n$ we have $(hg^{-1})_i = (hg^{-1})_{s(i)}$, or equivalently
$h_i^{-1} h_{s(i)} = g_i^{-1} g_{s(i)}$. %This already shows that the identity $\psi_g(s) = \psi_h(s)$ holds in the special case when $s$ is a transposition. 
For example, if $s$ is a cycle of length $p$, i.\,e. $s=(i_1, i_2, \ldots i_p)$ with $i_{k+1} = s(i_k)$ we get that
\begin{multline} \nonumber
 \psi_g(s) = \psi_g\left({\prod\limits_{k=1}^{p-1}(i_k, i_{k+1})}\right) = 
 \prod\limits_{k=1}^{p-1}\left(i_k, i_{k+1}\right)_{g_{i_k}^{-1} g_{i_{k+1}}} = 
 \prod\limits_{k=1}^{p-1}\left(i_k, i_{k+1}\right)_{h_{i_k}^{-1} h_{i_{k+1}}} = \psi_h(s).
\end{multline}
The proof for a general $s$ is almost the same.
Verification of the fact that $\psi$ and $\varphi$ are mutually inverse is also immediate.
\end{proof}

\subsection{Coxeter-like presentation for $S_n(G)$.}
The aim of this subsection is to show the following.
\begin{prop}\label{prop:Cox-Amalgam}
The group $S_n(G)$ defined in~\cref{prop:Q-pres} coincides with the group defined in the introduction by means of Coxeter-like presentation
(denote the latter group by $S_n^{C}(G)$).
\end{prop}

The idea of the proof is to construct a sequence of ''intermediate`` groups
\[S_n^C(G) \to S_n(1, G) \to \ldots \to S_n(n-1,G) \to S_n(G)\]
and then show that all the maps between these groups are isomorphisms.

\begin{df}
For $1\leq t\leq n-1$ consider the group $S_n(t, G)$ defined by the set of generators 
$\{(ij)_a \mid 1\leq i<j \leq n,\ j-i\leq t\}$ subject to the following relations:
\begin{align}
(ij)_a^2                      = &\, 1,                   &               \label{CoxMix1} \\
(ij)_a^{(jk)_b}               = &\, (jk)_{b'}^{(ij)_{a'}}, & \text{provided } ab = a'b',      \label{CoxMix2} \\
(ij)_a (kl)_b                 = &\, (kl)_b (ij)_a,       &               \label{CoxMix3} \\
(ij)_a ^ {(jk)_b}             = &\, (ik)_{ab}.           &               \label{CoxMix4}
\end{align}
In the above relations indices $i,j,k,l$ satisfy the inequalities ensuring that all the generators appearing in the formulae are well-defined.
\end{df}

In the case $t=1$ the relation~\eqref{CoxMix4} becomes vacuous and the remaining relations have the same form as \eqref{Cox1}---\eqref{Cox3}.
Thus, the map $S_n^C(G) \to S_n(1, G)$ defined by $s_i(g)\mapsto (i,i+1)_g$ is an isomorphism. 

On the other hand, in the case $t=n-1$ the obvious embedding of generators induces a map $S_n(n-1,G) \to S_n(G)$ 
 (we consider the latter group in the presentation of~\cref{prop:Q-pres}).
We claim that this map is an isomorphism. Indeed, define the map $\theta\colon S_n(G) \to S_n(n-1, G)$ in the opposite direction by the following formula:
\[\theta((ij)_a) = \left\{\def\arraystretch{1.2}%
  \begin{array}{@{}c@{\quad}l@{}}
    (ij)_a & \text{if $i < j$,}\\
    (ji)_{a^{-1}} & \text{if $j < i$.}\\
  \end{array}\right.\]
Only the fact that $\theta$ preserves the relation~\eqref{Q2} is not immediately obvious.
Fix three indices $i,j,k$ satisfying $i<j<k$ and let $p,q,r$ be any permutation of these indices.
One has to check that the equality $\theta((pq)_a^{(qr)_b}) = \theta((pr)_{ab})$ holds.
For example, the relation $\theta((jk)_a^{(ki)_b}) = \theta((ji)_{ab})$ can be proved as follows:
\[ \theta((jk)_a^{(ki)_b}) = (jk)_a^{(ik)_{b^{-1}}} = {(ik)_{b^{-1}}}^{(jk)_a} = (ij)_{b^{-1}a^{-1}} = \theta((ji)_{ab}).\]
In the above formula the second equality holds by~\eqref{CoxMix2} and the third one holds by~\eqref{CoxMix4} (in which both sides are conjugated by $(jk)_a$).
Verification of the other 5 cases is similar.

To finish the proof of~\cref{prop:Cox-Amalgam} it remains to show the following lemma.
\begin{lemma} \label{lem:coxeter-amalgam}
For each $1 \leq t \leq n-2$ the map $f_t \colon S_n(t, G)\to S_n(t+1, G)$ induced by the obvious embedding of generators is an isomorphism.
\end{lemma}
\begin{proof}
Let $(ij)_a$ be a generator of $S_n(t+1, G)$, define the map $g_t$ in the opposite direction as follows:
\[g_t((ij)_a) = \left\{\def\arraystretch{1.2}%
  \begin{array}{@{}c@{\quad}l@{}}
    (ij)_a & \text{if $j-i\leq t$,}\\
    (ik)_{a}^{(kj)_1} & \text{if $j-i=t+1$ for some $i<k<j$.}\\
  \end{array}\right.\]
The value of $g_t((ij)_a)$ does not depend on the choice of $k$.
Indeed, let $k'> k$ be another index lying between $i$ and $j$. 
By~\eqref{CoxMix3} we have the equality $(ik)_a^{(k'j)_1} = (ik)_a$ hence
\[(ik)_a ^{(kj)_1} = (ik)_a ^{\left((kk')_1 ^{(k'j)_1}\right)} = (ik)_a^{(kk')_{1} (k'j)_{1}} = (ik')_a ^{(k'j)_1}.\]

Obviously, the map $g_t$ would be the inverse to $f_t$ once we show that $g_t$ preserves the defining relations of $S_n(t+1, G)$.
Below we outline the proof of this fact for relation~\eqref{CoxMix2} and leave the verification for other relations as an exercise.

First consider the case $k-j\leq t$ and $j-i=t+1$. Choose some index $j'$ between $i$ and $j$.
Using the defining relations of $S_n(t,G)$ we get the following chain of equalities:
\begin{multline}\label{CoxMix2New} g_t((ij)_a^{(jk)_b}) = (ij')_{a} ^ {(j'j)_1 (jk)_b} = (ij')_a ^{ (jk)_b (j'j)_1 (jk)_b } = 
(ij')_a ^{(jk)_1 (j'j)_b (jk)_1} = \\ = (ij')_a ^{(j'j)_b (jk)_1} = (j'j)_{b'}^ {(ij')_{a'} (jk)_1} = 
 (j'j)_{b'}^ {(jk)_1 (ij')_{a'}} = \\ = (jk)_{b'}^{(j'j)_1 (ij')_{a'}} = (jk)_{b'}^{(ij')_{a'} (j'j)_1 (ij')_{a'}}
 = (jk)_{b'}^{(j'j)_1 (ij')_{a'} (j'j)_1} = g_t((jk)_{b'}^{(ij)_a}). \end{multline}
The proof in the case $k-j = t+1$ is similar:
\[g_t((ij)_a^{(jk)_b}) = (ij)_a ^ {(k'k)_1 (jk')_b (k'k)_1} = (ij)_a ^ {(jk')_b (k'k)_1} =
(jk')_{b'} ^ {(ij)_{a'} (k'k)_1} = (jk')_{b'} ^ {(k'k)_1 (ij)_{a'}} = g_t((jk)_{b'}^{(ij)_{a'}}).\]
Here $k'$ is some index lying between $j$ and $k$.
In the central equality we use either~\eqref{CoxMix2} or~\eqref{CoxMix2New} depending on whether $j-i\leq t$ or $j-i=t+1$.
\end{proof} 

\section{Comparison with extensions of type \texorpdfstring{$\mathfrak{H}_n(G)$}{Hn(G)}} \label{sec:Hnextensions}
We start by briefly reviewing the material of \S~1--3 of~\cite{Reh78}. 
Let $n \geq 3$ be a natural number. By definition, the group $H_n(G)$ is given by generators
$h_{ij}(u)$, $u\in G$, $i\neq j$ and the following relations (cf. \cite[H1--H5~of~\S~2]{Reh78}):
\begin{align}
h_{ij}(u) h_{ji}(u)                = &\, 1,                        &                     \tag{H1} \label{RH1} \\
h_{ij}(u) h_{ki}(u) h_{jk}(u)      = &\, 1,                        &                     \tag{H2} \label{RH2} \\
h_{ij}(u) h_{ik}(v) h_{ij}(u)^{-1} = &\, h_{ik}(uv) h_{ik}(u)^{-1},& \text{for } j\neq k \tag{H3} \label{RH3} \\
h_{ij}(u) h_{kj}(v) h_{ij}(u)^{-1} = &\, h_{kj}(vu) h_{kj}(u)^{-1},& \text{for } i\neq k \tag{H4} \label{RH4} \\
[h_{ij}(u), h_{kl}(v)]             = &\, 1.                        &                     \tag{H5} \label{RH5} \end{align}
There is a surjective group homomorphism $H_n(G) \twoheadrightarrow D_n(G)$ sending $h_{ij}(g)$ to $d_{ij}(g)$.
Here $d_{ij}(g)$ denotes the element of $G^n$ whose
 $i$-th component equals $g$, $j$-th component equals $g^{-1}$ and all other components are trivial.

By definition, {\it an extension of type $\mathfrak{H}_n(G)$} is an extension $H$ of $D_n(G)$ which is also a quotient of $H_n(G)$, i.\,e. 
 the one which fits into the following diagram.
\[ \xymatrix{ H_n(G) \ar@{->>}[rd] \ar@{->>}[d] & \\ H \ar@{->>}[r] & D_n(G)} \]
Any $\mathfrak{H}_n(G)$-extension is central, see~\cite[Proposition~2.3]{Reh78}.

\begin{comment}
We now recall the notion of \emph{an extension of type $\mathfrak{U}(G)$} introduced by Rehmann in~\cite[\S~1]{Reh78}.
Let $G$ be a group. Denote by $U(G)$ the group presented by generators $(u, v)$, $u,v\in G$ and relations:
\begin{align}
 ({}^{u} v, {}^{u}w) (u,  w) = &\, (uv, w), \tag{U1} \label{U1} \\
 (u,  vw) (v,  wu) (w,  uv)  = &\, 1.       \tag{U2} \label{U2}
\end{align}
There is a left action of $G$ on $U(G)$ given by ${}^x (u, v) = ({}^{x}u, {}^{x}v).$
Equivalently, $U(G)$ can be defined by the following relations, see~\cite[p.~8]{De76}.
\begin{align} 
 (u, v) (v, u)     = &\, 1,          \label{U4} \\
 (u, v) \ {}^v(u, w) = &\, (u, vw),  \label{U6} \\
 {}^u(v, w)(w, v)  = &\, (u, [v, w]).\label{U9} \end{align}
It is clear from the definition that $U(G)$ is a quotient of the nonabelian tensor square $G\otimes G$ and that
the map $(u, v) \mapsto [u, v]$ defines a $G$-equivariant map $U(G) \twoheadrightarrow [G, G]$.
Notice that in the literature one can find alternative notations for $U(G)$ which emphasize its
 connection with nonabelian tensor products e.\,g. $(G, G)$ or $G \mathbin{\widetilde{\wedge}} G$.
 
By definition, an {\it extension of type $\mathfrak{U}(G)$} is simply a quotient of $U(G)$
 that fits into the following commutative diagram with $G$-equivariant arrows.
\[ \xymatrix{ U(G) \ar@{->>}[rd] \ar@{->>}[d] & \\ U \ar@{->>}[r] & [G, G]} \]

There is a correspondence between extensions of type $\mathfrak{H}_n(G)$ and $\mathfrak{U}(G)$.
One can obtain a $\mathfrak{U}(G)$-extension from given $\mathfrak{H}_n(G)$-extension via restriction.
Indeed, let $H$ be an extension of type $\mathfrak{H}_n(G)$ and $k$ be arbitrary number $1\leq k\leq n$.
Denote by $\iota_k$ the inclusion map $[G, G] \to D_n(G)$ that inserts an element $g \in [G, G]$ at the $k$-th position.
Then the corresponding extension $U_H$ is simply the pull-back of $H$ with respect to $\iota_k$.

In other words, $U_H$ is simply the subgroup of $H$ generated by 
symbols $c_{kj}(u, v)=h_{kj}(u) \cdot h_{kj}(v) \cdot h_{kj}(vu)^{-1}$ 
 (here $j$ is any index not equal $k$, $c_{kj}(u,v)$ does not depend on $j$, see~\cite[Proposition~2.1]{Reh78}).
It can be shown that $U$ is an extension of type $\mathfrak{U}(G)$ with kernel~$\mathrm{Ker}(\pi_U)$ 
 isomorphic to $\mathrm{Ker}(\pi_H)$, see~\cite[Propositions~2.2--2.3]{Reh78}.
In~\cite[\S~3]{Reh78} Rehmann also describes the inverse construction which assign to every $\mathfrak{U}(G)$-extension
 $U\twoheadrightarrow [G, G]$ and a number $n\geq 3$ some extension $H_U$ of type $\mathfrak{H}_n(G)$ that fits into the pull-back diagram above.
% TODO: Is this correspondence one-to-one?
% TODO: Does $G\otimes G$ correspond to $H_n(G)$ via this construction (and vice-versa)?
\end{comment}

\subsection{Presentation for the subgroup $HS_n(G)$}
We now obtain a presentation of the subgroup $HS_n(G)$.
For every $i\neq j$ and $a\in G$ we define the element $h_{ij}(a) \in HS_n(G)$ as follows:
\begin{equation} \label{eq:h-def} 
h_{ij}(a) = (ij)_{a} \cdot (ij)_1. 
\end{equation}
It is not hard to show that $h_{ij}(a)$ form a generating set for $HS_n(G)$.
In fact, there is an explicit formula how an element of $HS_n(G)$ originally expressed through $(ij)_{a}$'s can be rewritten in terms of $h_{ij}(a)$.
Indeed, if $h$ lies in $HS_n(G)$ and is written as $\prod_{k=1}^N(i_k j_k)_{a_k}$ for some $i_k\neq j_k$ and $a_k\in G$ then it can be rewritten as follows:
\begin{equation} \label{eq:rp} \tag{$\tau$}
 h = \prod_{k=1}^N h_{\sigma_k(i_k), \sigma_k(j_k)}(a_k),\text{ where } \sigma_k=\prod_{s=1}^{k-1} (i_s j_s) \in S_n. \end{equation}

We briefly recall the notion of a {\it rewriting process} given in~\cite[\S~2.3]{MKS76}.
If $G$ is a group presented by generators $a_\nu$ and relations $R_{\mu}(a_{\nu})$ and $H$ is its subgroup with a generating set $J_i(a_\nu)$ then
 a {\it rewriting process for $H$} is a function which maps every word $u$ in alphabet $a_\nu$ to a word $v$ in alphabet $s_i$ such that
 $u$ and $v[s_i:=J_i]$ define the same element of $G$ whenever $u$ represents an element of $H$.

With this terminology, the mapping~\eqref{eq:rp} defined above is a rewriting process for the subgroup $HS_n(G)$.
Since it does not arise as a rewriting process corresponding to a coset representative function,
this process is {\it not} a Reidemeister rewriting process in the sense of~\cite[\S~2.3]{MKS76}.
However, it still satisfies the following two key properties of a Reidemeister rewriting process (cf. with (v) and (vi) of~\cite[\S~2.3]{MKS76}): 
\begin{itemize}
 \item if $U$ and $U^*$ are freely equal words in $(ij)_a$ then $\tau(U)$ and $\tau(U^*)$ are also freely equal words in $h_{ij}(a)$;
 \item if $U_1$ and $U_2$ are two words in $(ij)_a$ which define elements of $HS_n(G)$ then the words $\tau(U_1U_2)$ and $\tau(U_1) \tau(U_2)$ are equal.
\end{itemize}
Using these two properties and repeating the arguments used in the proof of~\cite[Theorem~2.8]{MKS76}
one can simplify the generic presentation given by~\cite[Theorem~2.6]{MKS76} and obtain the following.
\begin{lemma} \label{lm:h-gen}

 For $n\geq 3$ the group $HS_n(G)$ admits presentation on the generators $h_{ij}(a)$ with the following two families of defining relations:
 \begin{align}
  h_{ij}(a) = &\, \tau\left((ij)_a \cdot (ij)_1\right); \label{eq:tau1} \\
  \tau(KRK^{-1}) = &\, 1, \label{eq:tau2} 
 \end{align}  
 where $R$ varies over relations of~\cref{prop:Q-pres} and $K$ is any word in $(ij)_1$, $i\neq j$.
\end{lemma}

From~\eqref{eq:tau1} one immediately obtains the equality $h_{ij}(1)=1$, $i\neq j$.
Thus, if we denote by $\sigma$ the permutation corresponding to a word $K$ and let $S_n$ act on $h_{ij}(a)$'s in the natural way
 we will get that the word $\tau(KRK^{-1})$ is equivalent to ${}^{\sigma}\tau(R)$ (modulo relations $h_{ij}(1)=1$). 

Since the relations from~\cref{prop:Q-pres} are respected by the action of $S_n$,
 only relations of the form $\tau(R)=1$ are, in fact, needed for the presentation of $HS_n(G)$.
Writing down what $\tau(R)$ is for each of \eqref{Q1}--\eqref{Q4} we get the following.
 
\begin{prop} \label{prop:HSpres} For $n\geq 3$ the group $HS_n(G)$ admits presentation with generators $h_{ij}(a)$ and the following list of relations:
\begin{align}
h_{ij}(1)                     = &\, 1,              \tag{R0} \label{H0} \\
h_{ij}(a) h_{ji}(a)           = &\, 1,              \tag{R1} \label{H1} \\
h_{jk}(b) h_{ik}(a) h_{ij}(b) = &\, h_{ik}(ab),     \tag{R2} \label{H2} \\
[h_{ij}(a), h_{kl}(b)]        = &\, 1,              \tag{R3} \label{H3} \\
h_{ij}(a)^{-1}                = &\, h_{ij}(a^{-1}). \tag{R4} \label{H4}
\end{align}
\end{prop}

\subsection{A simpler presentation of $H_n(G)$}
The aim of this subsection is to obtain a presentation of $H_n(G)$ that is simpler than the original presentation: 4 instead of 5 relations that are also shorter. 
Our main result is the following.
\begin{prop} \label{prop:simpler} 
For $n\geq 3$ relations \eqref{H0}--\eqref{H4} imply \eqref{RH1}--\eqref{RH5}.
Moreover, for $n\geq 4$ relations \eqref{H0}--\eqref{H3} are equivalent to \eqref{RH1}--\eqref{RH5}.
\end{prop}
For $u, v\in G$ we define the following two symbols:
\[c_{ij}(u,v)=h_{ij}(u)h_{ij}(v)h_{ij}(vu)^{-1}, c'_{ij}(u,v)=h_{ij}(u)h_{ij}(v)h_{ij}(uv)^{-1}.\]
Notice that~\eqref{RH4} implies $h_{ij}(1)=1$ therefore $c_{ij}(u, u^{-1}) = c'_{ij}(u, u^{-1})$.
Our goal is to show that~\eqref{RH3} can be omitted from the definition of $H_n(G)$ provided $n \geq 4$.
\begin{lemma} \label{item-lem33} If one excludes relation~\eqref{RH3} from the list of defining relations for the group $H_n(G)$, $n\geq 3$,
 the following still remain true:
 \begin{lemlist}
\item \label{item-lem33-cntr} The elements $c_{ij}(u, u^{-1})$ are central in $H_n(G)$;
\item \label{item-lem33-comm} One has $c'_{ij}(u, v) = [h_{ij}(u), h_{kj}(v)]$, $k\neq i, j$;
\item \label{item-lem33-conj}  One has ${}^{h_{ij}(w)}c_{kj}'(u, v) = c_{kj}'(u, w)^{-1} c_{kj}'(u, vw)$, $k\neq i, j$;
\item \label{item-lem33-conj2} One has ${}^{h_{ij}(w)} c'_{ij}(u, v) = c'_{ij}(uw, v) c'_{ij}(w, v)^{-1}$.
 \end{lemlist}
\end{lemma}
\begin{proof}
First one shows using~\eqref{RH4} that $c_{ij}(u, u^{-1})$ centralizes $h_{kj}(v)$ (cf. with the proof of~\cite[Lemma~2.1(2)]{Reh78}).
Since in any group $[a, b]=1$ implies $[a^{-1}, b] = [a, b^{-1}] = [a^{-1}, b^{-1}] = 1$ we get that $c_{ij}(u, u^{-1})$ also centralizes $h_{kj}(v)^{-1} = h_{jk}(v)$ and
 $c_{ij}(u^{-1}, u)^{-1} = c_{ji}(u, u^{-1})$ centralizes both $h_{kj}(v)$ and $h_{jk}(v)$.
Together with~\eqref{RH2} and~\eqref{RH5} this implies that $c_{ij}(u, u^{-1})$ centralizes all the generators of $H_n(G)$ and hence lies in the center of $H_n(G)$.

The second and third assertions are straightforward corollaries of~\eqref{RH4}.
The fourth assertion follows from the third one:
\[{}^{h_{ij}(w)} c'_{ij}(u, v) = {}^{h_{ij}(w)} c'_{kj}(v, u)^{-1} = (c'_{kj}(v, w)^{-1} c'_{kj}(v, uw))^{-1} = c'_{ij}(uw, v) c'_{ij}(w, v)^{-1}.\]
\end{proof}

\begin{lemma}
If one excludes relation~\eqref{RH3} from the list of defining relations for the group $H_n(G)$, $n\geq 3$, the following statements are equivalent:
\begin{enumerate}
\item \label{item1} \eqref{RH3} holds;
\item \label{item4} one has $c'_{ij}(u, v)^{-1} = c'_{ij}(v, u)$;
\item \label{item5} symbols $c'_{ij}(u, v)$ do not depend on $i$;
\item \label{item5} one has $c'_{ij}(u, vw) = c'_{ij}(wu, v) \cdot c'_{ij}(uv, w)$;
\item \label{item6} one has $c'_{ij}(u, vu^{-1}) = c'_{ij}(uv, u^{-1})$.
\end{enumerate}
\end{lemma}
\begin{proof}
Implications $(1) \implies (2) \implies (3) \implies (4)$ are essentially contained in the proof of~\cite[Lemmas~2.1-2.2]{Reh78}.
Implication $(4) \implies (5)$ is trivial.

We now prove $(5) \implies (1)$. Notice that~\eqref{RH4} implies ${}^{h_{ij}(u)^{-1}}h_{ik}(v) = h_{ik}(u)^{-1} h_{ik}(vu)$, therefore
using~\cref{item-lem33-cntr} we get that ${}^{h_{ij}(u)}h_{ik}(v) = {}^{c_{ij}(u, u^{-1}) h_{ij}(u^{-1})^{-1}}h_{ik}(v) = h_{ik}(u^{-1})^{-1} h_{ik}(vu^{-1}).$
Thus, \eqref{RH3} is equivalent to the equality 
\[ h_{ik}(uv) h_{ik}(u)^{-1} = h_{ik}(u^{-1})^{-1} h_{ik}(vu^{-1}),\]
or what is the same
\[c'_{ik}(u^{-1}, uv) = h_{ik}(u^{-1}) h_{ik}(uv) =  h_{ik}(vu^{-1}) h_{ik}(u) = c'_{ik}(vu^{-1}, u).\]
It is clear that the last equality is an equivalent reformulation of $(5)$.
\end{proof}

\begin{proof}[Proof of~\cref{prop:simpler}]
The assertion of the proposition becomes evident once one notices the following:
\begin{itemize}
 \item \eqref{RH2} and \eqref{RH4} are implied by \eqref{H0}--\eqref{H2};
 \item \eqref{H0} follows from \eqref{RH4};
 \item \eqref{H2} follows from \eqref{RH1}, \eqref{RH2} and \eqref{RH4};
 \item \eqref{RH4} and \eqref{RH3} are equivalent in the presence of \eqref{H4};
 \item for $n\geq 4$ the third statement of the previous lemma follows from~\cref{item-lem33-comm}.
\end{itemize}
\end{proof}

\begin{cor} \label{cor:main} For $n \geq 3$ the group $HS_n(G)$ is the quotient of the extension $H_n(G)$ by~\eqref{H4} (in particular, 
 is an extension of type $\mathfrak{H}_n(G)$).
%The associated $\mathfrak{U}(G)$ extension $U_{HS_n(G)}$ is isomophic to the quotient of $U(G)$ modulo single family of relations
%\begin{equation} \nonumber (u, u) = 1, \ u\in G. \end{equation}
Moreover, we have the following natural isomorphism:
\begin{equation} \Ker(HS_n(G) \to D_n(G)) = \Ker(S_n(G) \to G \wr S_n) \cong \HH_2(G, \ZZ). \end{equation} \end{cor}
\begin{proof}
Most of the assertions follow from~\cref{prop:simpler} and~\cite[Proposition~5]{De76}.
One also has to use the formula $c_{ij}(u, u) = c_{ij}(u, u^{-1})$ (see~\cite[p.~87]{Reh78}).
\end{proof}

\section{Proof of Theorem~2} \label{sec:main}
Recall that for arbitrary set $X$ one defines the space $EX$ as the the simplicial set whose set of $k$-simplices $EX_k$
 is $X^{k+1}$ and whose faces and degeneracies are obtained by omitting and repeating components. 
For a group $G$ we denote by $\pi_G$ the canonical map $EG \to BG$ sending $(g, h) \in EG_1$ to $g^{-1}h \in BG_1$.
 
Now let $N$ be a group acting on $X$. We define two simplicial sets $U$ and $V$ as follows:
\[ U = \bigcup\limits_{n\in N} E(\Gamma_{n.-}) \subseteq E(X\times X),\ \ V = \bigcup\limits_{x,y\in X}E(N(x\to y)) \subseteq EN. \]
Here $\Gamma_{n.-}$ is the graph of the function $(x \mapsto nx)$ and $N(x\to y)$ denotes the subset of elements $n\in N$ satisfying $nx=y$.
With this notation the subset $N(x\to x)$ coincides with the stabilizer subgroup $N_x \leq N$.

\begin{lemma} \label{lm:quillen-a} The simplicial sets $U$ and $V$ are homotopy equivalent. \end{lemma}
\begin{proof} First, we define yet another simplicial set $W$ as follows.
Its $k$-simplices $W_k$ are matrices $\left(\begin{smallmatrix}x_0 & x_1 & \ldots & x_k&\\ n_0 & n_1 & \ldots & n_k \end{smallmatrix}\right)$,
 where $x_i\in X$ are $n_i\in N$ are such that all $n_i$'s act each $x_j$ in the same way, i.\,e. $n_ix_j = n_{i'} x_j$ for $0\leq i,i',j\leq k$. 
 The faces and degeneracies of $W$ are the maps of omission and repetition of columns.
 
 Now there are two simplicial maps $f\colon W\to U$, $g\colon W\to V$ whose action on $0$-simplicies is given by 
  $f\left(\begin{smallmatrix}x_0 \\ n_0\end{smallmatrix}\right) = (x_0, n_0x_0)$, 
  $g\left(\begin{smallmatrix}x_0 \\ n_0\end{smallmatrix}\right) = n_0$. 
 To prove the lemma it suffices to show that $f$ and $g$ are homotopy equivalences. 
 The proof for $f$ and $g$ is similar, let us show, for example, that $g$ is a homotopy equivalence.
 
 In view of Quillen theorem A (cf.\cite[ex.~IV.3.11]{Kbook}) it suffices to show that for each $p$-simplex $d \colon \Delta^p \to V$ the 
  pullback $g/(p, d)$ of $d$ and $g$ is contractible.
 The simplicial set $g/(p, d)$ can be interpreted as the subset of $\Delta^p \times E(X)$ whose set of $k$-simplices consists of pairs
  $(\alpha\colon \underline{k}\to \underline{p}, (x_0, \ldots, x_k)\in X(\alpha, d)^{k+1})$.
 Here $X(\alpha, d)$ is the subset of $X$ consisting of all $x$ for which $d_{\alpha(i)}x = d_{\alpha(j)}x$ for $0\leq i,j\leq k$.
 Notice that the set $X_d := X(id_{\underline{p}}, d)$ is nonempty and is contained in every $X(\alpha, d)$ (it even equals $X(\alpha, d)$ for surjective $\alpha$).
 Now choose a point $\widetilde{x}\in X_d$ and consider the simplicial homotopy \[H\colon \Delta^p \times EX \times \Delta^1 \to \Delta^p\times EX\] 
  between the identity map of $\Delta^p \times EX$ and
 the map $\Delta^p \times c_{\widetilde{x}}$, where $c_{\widetilde{x}}$ is the constant map. 
 More concretely, $H$ sends each triple $(\alpha\colon \underline{k} \to \underline{p}, (x_0, \ldots, x_k), \beta\colon \underline{k}\to\underline{1})$
 to $(\alpha, (x_0, \ldots, x_{i-1}, \widetilde{x}, \ldots, \widetilde{x}))$, where $i$ is the minimal number such that $\beta(i)=1$.
 By the choice of $\widetilde{x}$ the image of $H$ restricted to $g/(p, d)\times \Delta^1$ is contained in $g/(p, d)$, hence $g/(p, d)$ is contractible. 
\end{proof}

Now suppose that $X=H$ is also a group upon which $N$ acts on the left.
\begin{cor} \label{cor:ker-iso}
Consider the following two simplicial sets:
\[ S = \bigcup\limits_{h\in H} BN_h \subseteq BN,\ \ \ T = \pi_N(U) \subseteq B(H \times H).\]
There is an isomorphism $\theta\colon \Ker(\pi_1(S) \to N) \cong \Ker(\pi_1(T) \to H \times H).$
Moreover, the higher homotopy groups of $S$ and $T$ are isomorphic. 
\end{cor}
\begin{proof}
Consider the following two pull-back squares:
\[ \xymatrix{ V  \ar@{^{(}->}[r] \ar[d] \pullbackcorner & \ar[d]^{\pi_N} EN \\
              S \ar@{^{(}->}[r] & BN } \ \ \ 
   \xymatrix{ U  \ar@{^{(}->}[r] \ar[d] \pullbackcorner & \ar[d]^{\pi_{H \times H}} E(H \times H) \\
              T \ar@{^{(}->}[r] & B(H \times H)}\] 
The required isomorphism can be obtained from the homotopy long exact sequence applied to the left arrows of these diagrams.\end{proof}

Now let $G$ be a group. Set $N = G \wr S_n$, $H = G^n$ and consider the left action of $N$ on $H$ given by $(g, s) \cdot h = gh^{s^{-1}}$, $g, h\in G^n$, $s\in S_n$.
If one reorders the components of $BG^{2n}$ accordingly, the simplicial subset $T \subset BG^{2n}$ from the above corollary
 becomes precisely the preimage of $0$ under $h_n \colon BG^{2n} \to \ZZ[BG]$.

It is also easy to compute the map $\pi_1(S) \to N$. Indeed, van Kampen theorem~\cite[Theorem~2.7]{May99} asserts that
$\pi_1(S)$ is isomorphic to the free product of stabilizer subgroups $N_{h} \leq N$ amalgamated over pairwise intersections $N_h \cap N_{h'}$, $h, h\in H$.
For $h \in G^n$ the subgroup $N_h$ consists of elements $(g, s) \in N$ satisfying $gh^{s^{-1}} = h$, i.e. elements of the form $(hh^{-s^{-1}}, s)$.
Thus, $N_h\cong S_n$, $\pi_1(S)$ is isomorphic to the group $S_n(G)$ and the map $\pi_1(S) \to N$ coincides with the map $\mu$ defined in section~\ref{sec:QnG-def}.
In particular, for an abelian $G$ its kernel is generated by symbols $c_{ij}(u, v)$.

The following lemma gives a more concrete description of the isomorphism $\theta$ from~\cref{cor:ker-iso}.
\begin{lemma} \label{lem:concrete-formula}
 If $G$ is abelian then every generator $c_{ij}(u, v) \in \Ker(\pi_1(S)\to N)$ is mapped under $\theta$ to the homotopy class of the loop $f_{ij}(u, v)$, 
  which is defined as follows:
 \[f_{ij}(u,v) = p_{ij}(u) \circ p_{ij}(v) \circ p_{ij}(u^{-1} v^{-1}), \text{ where }
   p_{ij}(x)=(x[i], x[i]) \circ (x^{-1}[i], x^{-1}[j]).\] 
\end{lemma}
\begin{proof}
Fix a vector $g\in G^n$ and denote by $\gamma_{ij}(g, x)$ the path of length $4$ in $W$ which connects 
the following five points of $W_0$:
\[
\Big(\begin{smallmatrix} e    \\ (g; e) \end{smallmatrix}\Big),\,
\Big(\begin{smallmatrix} x[i] \\ (g; e) \end{smallmatrix}\Big),\,
\Big(\begin{smallmatrix} x[i] \\ (d_{ij}(x) g, (ij)) \end{smallmatrix}\Big),\,
\Big(\begin{smallmatrix} e    \\ (d_{ij}(x) g, (ij)) \end{smallmatrix}\Big),\,
\Big(\begin{smallmatrix} e    \\ (d_{ij}(x) g, e) \end{smallmatrix}\Big).\]
One can easily check that the image of $\gamma_{ij}(g, x)$ in $S$ under $\pi_N g$ is precisely the element $h_{ij}(x)$, while
$\pi_{H\times H} f$ sends $\gamma_{ij}(g, x)$ to the loop $p_{ij}(x)$ (here $f, g$ are the maps from the proof of~\cref{lm:quillen-a}).
\end{proof}

\begin{lemma} \label{lm:endotr} The only natural endotransformations of the functor $\HH_2(-, \ZZ)\colon \catname{Groups}\to \catname{Ab}$ 
 are morphisms of multiplication by $n \in \ZZ$.
\end{lemma}
\begin{proof}
 Denote by $\eta$ an endotransformation $\HH_2(-, \ZZ) \to \HH_2(-, \ZZ)$.
 When restricted to the subcategory of free finitely-generated abelian groups $\catname{Add}(\ZZ) \subseteq \catname{Ab}$ the second homology functor
 coincides with the second exterior power functor $A \mapsto \wedge^2A$. 
 
 Recall from~\cite[Theorem~6.13.12]{Ba96} that the category of quadratic functors is equivalent to the category of quadratic $\ZZ$-modules (see Definition~6.13.5 ibid.)
 The functor $A \mapsto \wedge^2A$ is clearly quadratic and corresponds to the quadratic $\ZZ$-module $0 \to \ZZ \to 0$ under this equivalence.
 Thus, we get that $\eta$ restricted to $\catname{Add}(\ZZ)$ coincides with the the morphism of multiplication by $n\in \ZZ$.
 
 Consider the group $\Gamma_k = \langle x_1, y_1, \ldots x_k, y_k \mid [x_1, y_1]\cdot \ldots \cdot [x_k, y_k] \rangle$ (the fundamental group of a sphere with $k$ handles).
 It is clear that the abelianization map $\Gamma_k \to \ZZ^{2k}$ induces an injection $\HH_2(\Gamma_k, \ZZ) \cong \ZZ \to \wedge^2\ZZ^{2k}$.  
 Consider the following diagrams.
  \[ \xymatrix{ \ZZ \ar@{^{(}->}[r] \ar[d]_{\eta_{\Gamma_k}} & \ar[d]^{n \cdot} \wedge^2\ZZ^{2k} \\
                \ZZ \ar@{^{(}->}[r]                          & \wedge^2\ZZ^{2k} } \ \ \ 
     \xymatrix{   \ZZ  \ar[r]^(.35){\chi} \ar[d]_{\eta_{\Gamma_k}} & \ar[d]^{\eta_G} \HH_2(G, \ZZ) \\
                  \ZZ  \ar[r]^(.35){\chi}                          & \HH_2(G, \ZZ)}  \]
 From the left diagram it follows that $\eta_{\Gamma_k}$ is also the morphism of multiplication by $n$.
 For every element $x\in \HH_2(G, \ZZ)$ there exist an integer $k$ and a map $\chi\colon \Gamma_k\to G$ sending the generator of $\HH_2(\Gamma_k, \ZZ)$ to $x$.
 From the right square we conclude that $\eta_G(x) = nx$, as claimed.
\end{proof}

\begin{proof}[Proof of~\cref{thm:main}]
\begin{comment}Factor $h_n$ as a composition of a trivial cofibration followed by a fibration: 
\begin{equation} \label{eq:fibr-repl} \xymatrix{BG^{2n} \ar@{^{(}->}[r] & E_{h_n} \ar@{->>}^(.45){ev_1 \circ \pi_2}[r] & \ZZ[BG]} \end{equation}
For example, we can define $E_{h_n}$ and the homotopy fiber $F_{h_n}$ via the usual path space construction
 (here we use the fact that $BG$ and $\ZZ[BG]$ are fibrant).
\[ \xymatrix{ E_{h_n}  \ar[r]_{\pi_2} \ar[d]_{\pi_1} \pullbackcorner & \ar[d]^{ev_0} \ZZ[BG]^I\\
              BG^{2n} \ar[r]_{h_n} & \ZZ[BG] } \ \ \ 
   \xymatrix{ F_{h_n}  \ar[r] \ar[d] \pullbackcorner & \ar[d]^{ev_1\,\circ\,\pi_2} E_{h_n} \\
              pt \ar[r]_{0} & \ZZ[BG]}  \]
\end{comment}              
Factor $h_n$ as a composition of a trivial cofibration followed by a fibration $BG^{2n}\to E_{h_n} \to \ZZ[BG]$ and
write down the starting portion of the long homotopy exact sequence for the latter.
If we denote by $K$ the kernel of the map $\nu\colon \pi_1(T) \to G^{2n}$ induced by the embedding $h_n^{-1}(0)=T \subseteq BG^{2n}$ we obtain the following commutative diagram:
\[ \xymatrix{1 \ar[r] & K       \ar[d]_{\psi} \ar[r]_{\beta} & \pi_1(T) \ar[d]_{\varphi} \ar[r]_{\nu}     & G^{2n} \ar[r] \ar[d]_{\cong} & \HH_1(G, \ZZ) \ar[r] \ar@{=}[d] & 1 \\
             1 \ar[r] & \HH_2(G, \ZZ) \ar[r]_{\alpha} & \pi_1(F_{h_n})           \ar[r]  & \pi_1(E_{h_n}) \ar[r]                       & \HH_1(G, \ZZ) \ar[r] & 1.}\]
We already know by Corollaries~\ref{cor:main} and~\ref{cor:ker-iso} that $K$ is naturally isomorphic to $\HH_2(G, \ZZ)$ provided $n\geq 3$.
Thus, it suffices to show that $\psi$ is an isomorphism. In view of~\cref{lm:endotr} we only need to consider the case when $G$ is abelian.

We are going to compute the image of each generator $c_{ij}(x, y) = x \wedge y \in \HH_2(G, \ZZ)$ under $\psi$.
In order to do this we need to obtain a more explicit description of the set of 1-simplices of $F_{h_n}$.
Since both $BG^{2n}$ and $\ZZ[BG]$ are fibrant $E_{h_n}$ and $F_{h_n}$ can be constructed using the path space construction, i.\,e. $E_{h_n} = BG^{2n} \times_{\ZZ[BG]} \ZZ[BG]^I$.
In this case the set of $1$-simplices of $F_{h_n}$ can be identified with the set of triples $(g, t, t') \in G^{2n}\times \ZZ[BG]_2 \times \ZZ[BG]_2$ 
 satisfying the following identites:
\[d_2(t) = h_n(g),\ \ d_1(t) = d_1(t'),\ \ d_0(t') = 0.\]
It will be convenient for us to perform all calculations inside the subset $F'\subset (F_{h_n})_1$ consisting of those triples for which $t'=0$ and $d_1(t)=d_0(t)=0$.
For shortness we write down an element of $F'$ as $(g, t)$.
It is easy to check that the image of the loop $p_{ij}(x)$ under the natural map $T \to F_{h_n}$ corresponds to the pair
$((e, d_{ij}(x)), t(x)),$ where $t(x)=-(x, x^{-1}) + (e, e) - (x^{-1}, e) + (e, x^{-1})$.

We want to devise a concrete formula for the element of $F'$ homotopic to the concatenation of two elements $(g_1, t_1), (g_2, t_2) \in F'$.
It is easy to check that $(g_1g_2, t)$, where $t$ is the $1$-st face of the filler $f$ for the $3$-horn $(t_2, ,t_1, h_n(g_1, g_2))$, is the desired element.
By Moore's theorem one can find a filler for any horn $(p_0, ,p_2, p_3)$ in a simplicial abelian group via the formula
 $f = s_0 p_0 - s_0 s_1 d_2 p_0 - s_0 s_0 d_1 p_0 + s_0 s_0 d_2 p_0 + s_1 p_2 + s_2 p_3 - s_1 p_3$. 
Substituting concrete values of $p_i$ into this formula and using the fact that $d_1t_2=0$ we get the following expression for $t$:
\begin{multline} \nonumber t = d_1f = p_0 - s_1 d_2 p_0 - s_0 d_1 p_0 + s_0 d_2 p_0 + p_2 - p_3 + s_1 d_1 p_3 = \\
             = t_1 + t_2 - h_n(g_1, g_2) - s_1 h_n(g_2) + s_0 h_n(g_2) + s_1 h_n (g_1g_2). \end{multline}
Applying the above formula twice we get that the image of $c_{ij}(u,v)$ under $\varphi\beta$ equals $((e, e), t)$, where
\begin{multline} \label{eq:formula-t} t = 2(e, e)-(x^{-1}, e) +(xy, x^{-1}y^{-1}) -(xy, e) -(x, x^{-1}) -(y, y^{-1}) -\\  -(e, x^{-1}y^{-1}) +(e, x^{-1}) +(x, y) +(x^{-1}, y^{-1}) -(e, y) -(xyx^{-1}y^{-1}, e) +(y, e).\end{multline}

The map $\alpha$ can be described as the Dold---Kan isomorphism $\HH_2(G, \ZZ) \cong \pi_2(\ZZ[BG])$ followed by the boundary map $\partial\colon \pi_2(\ZZ[BG]) \to \pi_1(F_{h_n})$.
It is not hard to show that $\partial$ maps each $c \in \pi_2(\ZZ[BG])$ to $((e,e),-c)$ (see~\cite[p.~29]{GJar09}).
Thus, we get that $\psi$ maps $x \wedge y$ to $-t$, where $t$ is given by~\eqref{eq:formula-t}.

On the other hand, the generator $x \wedge y$ of $\HH_2(G, \ZZ)$ corresponds to the 2-cycle $c = (x, y) - (y, x)$ (cf.~\cite[(14), p.~582]{Mi52}) and the corresponding
element of $\pi_2(\ZZ[BG])$ is simply the normalized 2-cycle of $c$:
\begin{equation} \nonumber c' = c - s_0d_0c - s_1d_1c + s_1d_0 = (x, y) - (y, x) - (x, e) +(y, e) + (e, x) - (e, y). \end{equation}
It it not hard to check that $t-c'$ is the image of the following $3$-chain under the differential $d_0-d_1+d_2-d_3$:
\[(xy, x^{-1}, y^{-1}) - (y, x, x^{-1}) + s_1 s_1(x^{-1} + xy - x - y) + s_0 s_0(x^{-1} - x - x^{-1}y^{-1}).\]
Thus we obtain that $[t] = -[c']$, hence $\psi$ is an isomorphism, as claimed.
\end{proof}

\appendix\section{Simplicial Dold---Serre fibrations}
The aim of this appendix is to show that the map $h_\infty\colon X^{\infty}\to \ZZ[X]$ mentioned in the introduction is a quasifibration.

We start with the definition of the simplicial analogue of the so-called ''Dold---Serre fibration``, i.\,e.
a map posessing ''weak covering homotopy property`` (also called ''delayed homotopy lifting property``), cf.~\cite[\S~1.4.4]{KSz14}.
We show in~\cref{lm:dhlp,lm:weak-equiv} that the map $h_\infty$ satisfies this property and that every map with this property is a quasifibration.
 
\begin{df} \label{df:dhlp}
 Let $C$ be a subclass of the class of acyclic cofibrations in the category of simplicial sets.
 We say that a map of simplicial sets $p\colon E \to B$ satisfies {\it weak homotopy covering property} with respect to $C$ if 
  for every commutative square
\begin{equation} \xymatrix{ 
U \ar[r]^{g}  \ar@{^{(}->}_{i}[d] & E \ar^{p}[d] \\
V \ar[r]_{F} \ar@{-->}[ru]^{\widetilde{F}} & B}
\label{eq:plp-def} \end{equation}
for which $i \in C$ there exists a map $\widetilde{F}$ such that the lower triangle commutes strictly and the upper one commutes up to a fiberwise homotopy $H$
 (i.\,e. a homotopy $H$ such that $pH \colon U \times I \to B$ coincides with the composite $U \times I \xrightarrow{\pi} U \xrightarrow{pg} B$).
\end{df}

Denote by $C_{pr}$ the class consisting of all inclusions $i\colon U\hookrightarrow V$ of finite polyhedral simplicial sets 
 for which there exists a deformation retract (i.\,e. a map $r\colon V\to U$ homotopic to $\mathrm{id}_V$).
\begin{lemma} \label{lm:dhlp} For a fibrant and connected simplicial set $X$ the map $h_\infty \colon X^\infty \to \ZZ[X]$ satisfies the weak covering homotopy property with respect to the class $C_{pr}$.
\end{lemma}
\begin{proof}
 Let $i_0$ be a map from $C_{pr}$ and $r$ be the corresponding retraction.
 We first reduce the problem to the special case when $g$ is the constant map at the basepoint of $X^\infty$ (which we denote by $0$).
 Indeed, let $F, g$ be arbitrary maps as in~\eqref{eq:plp-def}.
 Since $U$ is finite the image of $g$ is contained in $X^{2N} \subset X^\infty$.
 If now $\widetilde{F}_0$ is a lifting in the diagram similar to~\eqref{eq:plp-def} in which the top map equals $0$ and the bottom one is $F_0 = F - h_\infty gr$
  then $\widetilde{F} = gr \times \widetilde{F}_0$ is the desired lifting of $F$.
 %For any maps $f\colon A \to X^\infty$ and $g\colon A \to X^{2N}$ one can define their product $g\times f$ by $(g \times f)(a) = (g(a); f(a)) \in X^\infty$. Clearly, then .
  
 Now choose a contractible fibrant simplicial set $W$ which maps surjectively onto $X$ (e.\,g. take $W$ to be the path space fibration).
 Since $p \colon W \twoheadrightarrow X$ is surjective, the associated map between free simplicial abelian groups is a Kan fibration. %Goerss--Jardine Lemma III.2.8
 Since $i_0$ is acyclic we can choose a lifting $\widetilde{F}$ in the following diagram
\begin{equation} \nonumber \xymatrix{ 
U \ar[r]^{0}  \ar@{^{(}->}_{i_0}[d] & \ZZ[W] \ar@{->>}[d]  \\
V \ar[r]_{F} \ar[ru]_{\widetilde{F}} & \ZZ[X]}
\end{equation}

 It is clear that the image of $\widetilde{F}$ is contained in the simplicial subgroup $\ZZ[W]_0 \leq \ZZ[W]$ which, by definition, consists of linear combinations $\sum_i n_i w_i$ for which $\sum_i n_i = 0$.
 By~\cite[Lemma~9.1]{Po17} the canonical map $\ZZ[\underline{\Hom}(V, W)]_0 \to \underline{\Hom}(V, \ZZ[W]_0)$ is surjective, hence
  the map $\widetilde{F}$ can be lifted along $h_W$ in such a way that the bottom triangle in following diagram commutes strictly.
\begin{equation} \nonumber \label{eq:W-lift} \xymatrix{ 
U \ar[r]^{0}  \ar@{^{(}->}_{i_0}[d]                               & W^\infty \ar[d]^{h_W}  \\
V \ar[r]_{\widetilde{F}} \ar@{-->}[ru]^{G} & \ZZ[W]}
\end{equation}
 Notice that the image of $Gi_0$ is contained in the fiber $h_W^{-1}(0)$.
 It is easy to see that $h_W^{-1}(0)$ is contractible (the contracting homotopy for $W^\infty$ can be restricted to $h_W^{-1}(0)$).
 We get that $Gi_0$ is fiberwise homotopic to $0$ hence the composite map $V \xrightarrow{G} W^\infty \to X^\infty$ is the desired lifting of $F$. \end{proof}

\begin{comment} 
The assertions of the following lemma are straightforward and are given without proof.
\begin{lemma} \label{lem:topo-facts} Let $p\colon (E, e) \to (B, b)$ be a map of pointed topological spaces.
Let $p^{-1}(b) \hookrightarrow F_{p}(b) \subseteq E\times_B B^I$ be the inclusion map of the fiber of $p$ into the homotopy fiber.
Denote $k$-th relative homotopy group (or set) $\pi_k(F_p(b), p^{-1}(b), e))$ by $G_k$.
\begin{lemlist}
 \item Maps of triples $(D^k, S^{k-1}, pt) \to (F_p(b), p^{-1}(b), e)$ are in one-to-one correspondence with commuting diagrams of the form
   \begin{equation} \xymatrix{ 
    D^k \ar@{^{(}->}_{i_0}[d]  \ar_{a}[rr] & & E \ar^{p}[d]  \\
    D^k\times I \ar^{\pi}[r] & (D^k \times I) / J \ar^(.7){A}[r] & B}
    \label{eq:plp} \end{equation}
    Here $J$ denotes $(S^{k-1} \times I) \cup (D^k \times \{1\})$. In the sequel we denote such a diagram by $(a, A)$.
 \item Two diagrams $(a_0, A_0)$ and $(a_1, A_1)$ represent the same element of $G_k$ iff there exists a ''diagram homotopy`` that connects them.
     By a diagram homotopy we mean a family of maps $(a_t, A_t)$ continously depending on $t\in [0, 1]$ such that $pa_t = A_t \pi i_0$ holds for all $t$.
 \item \label{item:continue} Let $(a, A)$ be a diagram and $a'$ be any other map 
     homotopic to $a$ via some homotopy $H$ such that $H(S^{k-1}\times I) \subseteq p^{-1}(b)$.
     Then there exists a map $A'$ such that $(a', A')$ is a diagram homotopic to $(a, A)$.
 \item \label{item:weaker} 
     Assume that in the diagram~\eqref{eq:plp} there exists a diagonal map $\widetilde{A} \colon D^k\times I \to E$ 
     such that the bottom triangle is commutative and the upper is commutative up to a fiberwise homotopy.
     Then the element of $G_k$ given by $(a, A)$ is trivial.
\end{lemlist}
\end{lemma}

Denote by $H$ the fiberwise homotopy between $a$ and $\widetilde{A}i_0$.
 Consider the following family of diagrams:
 \[ a_t(x) = \left\{\def\arraystretch{1.2}%
  \begin{array}{@{}c@{\quad}l@{}}
    H(x, 2t)               & 0\leq t\leq \frac{1}{2}\\
    \widetilde{A}(x, 2t-1) & \frac{1}{2}\leq t \leq 1\end{array}\right.;\  \
  A_t(x, s) = \left\{\def\arraystretch{1.2}%
  \begin{array}{@{}c@{\quad}l@{}}
    A(x, s)                & 0\leq t\leq \frac{1}{2} \\
    A(x, s + (2t-1)(1-s)) &  \frac{1}{2}\leq t \leq 1
  \end{array}\right.\]
 It is clear that $a_0=a$ and $A_0=A$ while the image of $a_1$ is contained in $p^{-1}(b)$.

\begin{lemma} \label{lm:weak-equiv} Let $p\colon E \to B$ be a map of simplicial sets satisfying the weak covering homotopy property with respect to the class $C_{pr}$.
 Then the geometric realization of $p$ is a quasifibration, i.\,e. for every point $b \in B_0$ the inclusion $|p|^{-1}(b) \hookrightarrow F_{|p|}(b)$ is a homotopy equivalence. \end{lemma}
\begin{proof}

 It suffices to show that $\pi_k(F_{|p|}(b), |p|^{-1}(b))$ are all trivial for $k \geq 1$.
 Consider a diagram $(a, A)$ of the form~\eqref{eq:plp} (with the map $|p|$ in the right hand side). 
 In view of the previous lemma it suffices to construct a lifting $\widetilde{A}$ satisfying the requirements of~\cref{item:weaker}.
 
 The idea of the proof is to approximate $(a, A)$ with a homotopic pair of geometric realizations of simplicial maps and then invoke the weak covering homotopy property. 
 The key ingredient in the proof below is the simplicial approximation theorem~\cite[Theorem~4.7]{Jar04}.
 For brevity we denote the subdivision functor $\sd^m\sd_*(-)$ from its statement by $s^m(-)$ and the canonical natural transformation $s^m X \to X$ by $\delta^m$.
 
 We proceed in a number of steps.
 Denote the restriction of $a$ to $S^{k-1}$ by $a_0$.
 Notice that the image of $a_0$ is contained in $|p^{-1}(b)|$.
 Using the approximation theorem we find a simplicial map $a_0'\colon s^m (S^{k-1}) \to p^{-1}(b)$ such that $a|\delta^m| \cong |a'_0|$.     
 
 We can find a map $a'\colon D^k\to |E|$ extending $|a'_0|$ and homotopic to $a$ via some homotopy $H$ satisfying $H(S^k\times I) \subseteq p^{-1}(b)$.
 
 Invoking the approximation theorem once again (with the initial condition specified by $|a'_0|$) we find $q>m$ and $a''\colon s^{q}(D^k) \to E$
 such that $a'|\delta^{q}| \cong |a''|$ rel $S^k$.
 Using \cref{item:continue} we extend the homotopy $a |\delta^q| \cong a'|\delta^q| \cong |a''|$ to a homotopy of diagrams $(a|\delta^q|, A|\delta^q|) \cong (|a''|, A')$.
 
 Applying the approximation theorem to $A'$ with the initial condition on the boundary $\partial (D^k \times I) = D^k\times \{0\} \cup J$ specified by 
 $pa''$ and $const_b$, respectively, we find an integer $r>q$ and a map $A''\colon s^{r}(D^k \times I) \to B$ such that
 $A'|\delta^r|\cong |A''|$ rel $\partial (D^k \times I)$.
 
 Thus, we have obtained the following commutative diagram of simplicial sets for which $(|a''|, |A''|)$ is a diagram homotopic to $(a, A)$.
   \begin{equation} \nonumber \xymatrix{ 
    s^{r}(D^k)          \ar[r]^(0.7){a''\gamma^{r-q}}  \ar@{^{(}->}_{s^{r}(i)}[d] & E \ar[d]^{p}  \\
    s^{r}(D^k \times I) \ar[r]_(0.7){A''}  \ar@{-->}[ru]_{\widetilde{A}} & B}
   \end{equation}
 By functoriality of the subdivision functor the map $s^{r}(i)$ posesses a deformation retract hence there exists the desired lifting $\widetilde{A}$.
 \end{proof}
\end{comment}
 
%\end{appendices} 

\printbibliography

\end{document}