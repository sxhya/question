\documentclass[oneside, 12pt]{amsart}
\usepackage{amscd, amsmath, amssymb, amsthm, amsfonts, amstext, geometry, verbatim, enumitem, graphicx, mathtools, xfrac, microtype, nameref, thmtools}
\usepackage[breaklinks=true]{hyperref}
\usepackage[capitalize]{cleveref}
\usepackage[hyperref=true, backend=bibtex, firstinits=true, citestyle=numeric-comp, sortlocale=en_US, url=false, doi=false, eprint=true, maxbibnames=4]{biblatex}            
\usepackage[notref,notcite]{showkeys}
\usepackage[matrix,arrow,curve]{xy}

\addbibresource{question.bib}
\renewbibmacro*{volume+number+eid}{\ifentrytype{article}{\- \iffieldundef{volume}{}{Vol.~\printfield{volume},}\iffieldundef{number}{}{ No.~\printfield{number},}}}
\renewbibmacro{in:}{\ifentrytype{article}{}{\printtext{\bibstring{in}\intitlepunct}}}
\newbibmacro{string+doi}[1]{\iffieldundef{doi}{\iffieldundef{url}{#1}{\href{\thefield{url}}{#1}}}{\href{http://dx.doi.org/\thefield{doi}}{#1}}}
\DeclareFieldFormat[article, inproceedings, inbook, book, thesis]{title}{\usebibmacro{string+doi}{\mkbibquote{#1}}}
\renewcommand*{\bibfont}{\footnotesize}

\newlist{lemlist}{enumerate}{1} \setlist[lemlist]{label={\rm(\arabic{lemlisti})}, ref=\thelemma.(\arabic{lemlisti}),noitemsep} \Crefname{lemlisti}{Lemma}{Lemmas}

\theoremstyle{plain}

\newtheorem{thm}{Theorem}
\Crefname{thm}{Theorem}{Theorems}
\numberwithin{equation}{section}

\newtheorem{lemma}{Lemma}
\numberwithin{lemma}{section}
\Crefname{lemma}{Lemma}{Lemmas}

\newtheorem{cor}[lemma]{Corollary}
\Crefname{cor}{Corollary}{Corollaries}

\newtheorem{prop}[lemma]{Proposition}
\Crefname{prop}{Proposition}{Propositions}

\newtheorem*{thm*}{Theorem}
\newtheorem*{lemma*}{Lemma}

\theoremstyle{remark}

\newtheorem{rem}[lemma]{Remark}
\Crefname{rem}{Remark}{Remarks}

\theoremstyle{definition}

\newtheorem{df}[lemma]{Definition} \Crefname{df}{Definition}{Definitions}
\newtheorem{example}[lemma]{Example} \Crefname{example}{Example}{Examples}

\newcommand{\catname}[1]{{\normalfont\textbf{#1}}}
\DeclareMathOperator{\HH}{H}
\DeclareMathOperator{\colim}{colim}
\DeclareMathOperator{\Ker}{Ker}
\DeclareMathOperator{\Hom}{Hom}
\DeclareMathOperator{\sd}{sd}
\DeclareMathOperator{\hh}{\overline{{\it h}}}
\newcommand{\ZZ}{\mathbb{Z}}
\newcommand{\pushoutcorner}[1][dr]{\save*!/#1+1.2pc/#1:(1,-1)@^{|-}\restore}
\newcommand{\pullbackcorner}[1][dr]{\save*!/#1-1.2pc/#1:(-1,1)@^{|-}\restore}

\title{Amalgams of symmetric groups and $\mathfrak{H}_n$-extensions}
\keywords {$\mathfrak{U}(G)$-extensions, $\mathfrak{H}_n(G)$-extensions, amalgamated products, van Kampen theorem,
 {\em Mathematical Subject Classification (2010):} 55Q05, 20B30, 20E06, 20J06}

\author {Sergey Sinchuk}

\date {\today}

\begin{document}

\begin{abstract} By amalgamating copies of symmetric group $S_n$ we construct a central extension $S_n(G)$ of a large subgroup of the wreath product $G \wr S_n$.
We then establish a connection between this group and the theory of $\mathfrak{U}(G)$ and $\mathfrak{H}_n(G)$-extensions developed by K.~Dennis and U.~Rehmann in 1970's.
We also give a topological application of this construction by describing homotopy invariants of blah-blah-blah...
 %the fundamental group of the fiber of the ``alternating`` map $BG^{2n}\to \mathbb{Z}[BG]$.
\end{abstract}

\maketitle

\section{Introduction}
The aim of this note is to establish a surprising connection between symmetric groups and (certain quotients of) nonabelian tensors squares.

Let $G$ be a group. Denote by $D_n(G)$ the subgroup of $G^n$ consisting of vectors
 $(g_1,\ldots, g_n)$ for which the product $g_1\cdot \ldots \cdot g_n$ lies in the derived subgroup $[G, G]$.
Recall from Rehmann's paper~\cite{Reh78} that the universal {\it extension of type $\mathfrak{H}_n$} (denoted $H_n(G)$) 
 is, by definition, a certain (central) extension of $D_n(G)$ explicitly presented by generators and relations 
 (see section~\ref{sec:Hnextensions} below).
The extension $H_n(G)$ and its ''antisymmetrization`` $H^\wedge_n(G)$ can be characterised by the property that 
 they fit into the following pullback squares:
\[ \xymatrix{ G \mathbin{\widetilde{\wedge}} G \pullbackcorner \ar@{^{(}->}[d] \ar@{->>}[r] & [G, G] \ar@{^{(}->}[d]^{\iota_1} \\ H_n(G) \ar@{->>}[r] & D_n(G)} \qquad
   \xymatrix{ G \wedge G \pullbackcorner \ar@{^{(}->}[d] \ar@{->>}[r] & [G, G] \ar@{^{(}->}[d]^{\iota_1} \\ H^\wedge_n(G) \ar@{->>}[r] & D_n(G)} \]
Here $G \wedge G$ denotes the nonabelian exterior square of $G$ and  
 $G \mathbin{\widetilde{\wedge}} G$ is the a more modern notation for the Dennis group $(G, G)$ defined in~\cite{De76} (also denoted $U_G$ by Rehmann in \cite{Reh78}).
The map $\iota_1$ in the above diagrams is defined by $\iota_1(g) = (g, 1,\ldots, 1)$.

Consider the family of copies of the symmetric group $S_n$ where each copy is indexed by an $n$-tuple $(g_1, \ldots, g_n) \in G^n$.
Denote by $S_n(G)$ the free product of groups from this family amalgamated over single family of relations of the form
$s_g$ = $s_h$, where $s\in S_n$, $g, h \in G^n$ are such that $hg^{-1}$ is fixed by the natural action of $S_n$ on $G^n$
(here we denote by $s_g$ a permutation $s$ embedded into the copy with index $g$).
 
Our first main result is the following theorem. 
\begin{thm} \label{thm:summary} The group $S_n(G)$ is isomorphic to $S_n \ltimes H_n^\wedge(G)$. \end{thm}
The proof of Theorem 1 proceeds as follows: 
The presentation of Theorem is similar in spirit (although much simpler) than the presentation

Unsurprisingly, the presentation from Theorem~1 is obtained via van Kampen theorem and 
was discovered during the attempt to answer some purely topological question, which we describe in detail below.

For a pointed simplicial set $(X, x_0)$ one denotes by $\ZZ[X]$ the associated free simplicial abelian group.
Define the ''alternating`` map of simplicial sets $h_n\colon X^{2n} \to \ZZ[X]$ by the identity:
 \[h_n(x_1, x_2, \ldots x_{2n}) = \sum\limits_{i=1}^{2n}(-1)^ix_i.\]
Denote by $X^\infty$ the colimit of $X^n$ with respect to the natural embeddings $X^{n} \hookrightarrow X^{n+1}$
Taking colimit of $h_n$ we get the map $h_\infty = \mathrm{colim}_n(h_n) \colon X^\infty \to \ZZ[X]$.

Based on a recent result of S.~Podkorytov one can deduce that this map is a quasifibration provided $X$ is fibrant and connected.
In particular, the natural map $h^{-1}_\infty(0) \to F_{h_\infty}$ between the fiber and the homotopy fiber of $h_\infty$ over $0$ is a weak equivalence, see~\cref{lm:weak-equiv}.
\begin{comment}
For finite $n$ this no longer remains true, however one may expect that the homotopy groups of $h^{-1}_n(0)$ and $F_{h_n}$ are still
 isomorphic in small degrees.
S. Podkorytov conjectured that the natural map $h^{-1}_n(0) \to F_{h_n}$ is $c_n$-connected where $c_n$ tends to infinity as $n\to \infty$.
The main goal of this note is to show the following result which confirms this conjecture in one special case.
\end{comment}
\begin{thm} \label{thm:main} If $X=BG$ is the classifying space of a group $G$ then for $n\geq 3$
 the natural map $h_n^{-1}(0) \to F_{h_n}$ induces an isomorphism of fundamental groups
 $\pi_1(h^{-1}_n(0))\cong \pi_1(F_{h_n})$. \end{thm}

The proof of Theorem 2 goes as follows. ???
 
In the course of the proof we also find a presentation of the universal $\mathfrak{H}_n(G)$-extension 
 that is simpler than the original presentation introduced by Rehmann in~\cite{Reh78}, see~\cref{prop:simpler}.

\subsection{Acknowledgements}
I wish to thank S.~Podkorytov for suggesting the problem and also for his kind permission to use some of his unpublished results.
I am also grateful to S.~O.~Ivanov and V.~Isaev for their numerous helpful comments and interest in this work.

\section{The extension $S_n(G)$} 
\subsection{The definition of $S_n(G)$} \label{sec:QnG-def}
Let $G$ be a group. Recall that the {\it wreath product} is, by definition,
 the semidirect product $G^n \rtimes S_n$, in which $S_n$ acts on $G^n$ on the right by permuting its factors.

Consider the family $\{{S_n}^{(g)}\}_{g\in G^n}$ of isomorphic copies of $S_n$ indexed by elements of $G^n$ and let $F$ be the free product of groups from this family.
For $s\in S_n$ and $g\in G^n$ we denote by 
$s_{g}$ the image of $s$ in $F$ under the canonical map $S_n^{(g)} \to F$.

\begin{df} We define the group $S_n(G)$ to be the quotient of $F$ modulo the following single family of relations:
\begin{equation} \label{eq:main_rel} s_g = s_h, \text{ where $s$ fixes $h g^{-1}$.} \end{equation} \end{df}

We continue to denote by $s_g$ the image of the element $s_g\in F$ under the canonical map $F\to S_n(G)$.

\begin{rem}
Expanding the definition of the semidirect product we get that
\begin{equation} \nonumber s^g = (1_{G^n}, s)^{(g, 1)} = (g^{-1}, s) (g, 1) = (g^{-1} g^{(s^{-1})}, s). \end{equation}
therefore the equality $s^g = s^h$ holds in $G \wr S_n$ iff
$g^{-1} g^{(s^{-1})} = h^{-1} h^{(s^{-1})}$, or what is the same, iff $s$ fixes $hg^{-1}$.

The last statement immediately implies that the map $\mu \colon S_n(G) \to G \wr S_n$ given by $\mu(s_g) = s^g = (g^{-1}g^{(s^{-1})}, s)$
is well-defined. We will see shortly that $\mu$ is, in fact, a {\it crossed module} in the sense of~\cite[\S~2.2]{BHS11}.
\end{rem}

Observe from the definition of $S_n(G)$ that there is a split exact sequence.
\begin{equation} \label{eq:ex-seq} \xymatrix{1 \ar[r] & \mathrm{Ker}(\pi) \ar[r] &  S_n(G) \ar@<0.5ex>[r]^{\pi} & S_n \ar[r] \ar@{-->}@<0.5ex>[l]^{\iota(1)} & 1,} \end{equation}
Here the map $\pi = \pi_{S_n} \circ \mu$ removes subscript $g$ from each $s_g$ and the section $\iota(1)$ sends $s$ to $s_1$, where $1$ is the identity element of $G^n$.
Thus, if we denote $\mathrm{Ker}(\pi)$ by $HS_n(G)$ we get a decomposition $S_n(G) = HS_n(G) \rtimes S_n$.

\subsection{A crossed module structure on $S_n(G)$}
Recall from~\cite[\S~2.2]{BHS11} that a {\it crossed module} is a morphism of groups $\mu\colon M\to N$ together with a right action of $N$ on $M$ 
compatible with the conjugation action of $N$ on itself, i.\,e.
\begin{equation} \label{eq:precrossed} \tag{CM1} \mu(m^n) = \mu(m)^n \text{ for all $n \in N$, $m \in M$}, \end{equation}
which also satisfies the following identity called {\it Peiffer identity}:
\begin{equation} \label{eq:Peiffer} \tag{CM2} m^{m'} = m^{\mu(m')} \text{ for all $m, m' \in M$}.\end{equation}
In our situation, we let $G \wr S_n$ act on $S_n(G)$ by 
\begin{equation} \label{eq:action} (s_g)^{(h, t)} = {s^t}_{(gh)^t}, \text{ for $s, t \in S_n$ and $g, h \in G^n$.} \end{equation}
The goal of this subsection is to prove the following.
\begin{prop} \label{thm:cms} For $n \geq 3$ the map $\mu \colon S_n(G) \to G \wr S_n $ is a crossed module. \end{prop}
From the fact that $\mu$ is a crossed module one can deduce that $\Ker(\mu)$ is a central subgroup of $S_n(G)$ and $\mathrm{Im}(\mu)$ is a normal subgroup of $G \wr S_n$.
It is not hard to show that the group $S_2(G)$ is isomorphic to a free product of copies of $S_2$ (whose center is always trivial).
Therefore, the requirement $n\geq 3$ in the statement of~\cref{thm:cms} is essential.

Verification of the fact that formula~\eqref{eq:action} gives a well-defined action of $G\wr S_n$ on $S_n(G)$ 
 that satisfies~\eqref{eq:precrossed} is lengthy but straightforward. Let us show that~\eqref{eq:Peiffer} holds. 
It suffices to verify Peiffer identities only for the generators of $S_n(G)$, for which it takes the form:
\begin{equation} \label{eq:Peiffer-gen} t^{-1}_h s_g t_h = {s^t}_{(gh^{-1})^t\cdot h} \text{ for all $s, t \in S_n$, $g, h\in G^n$.}\end{equation}
If we act on both sides of the above formula by $(h^{-1}, 1) \in G \wr S_n$ we obtain the equality
$t^{-1}_1 s_{gh^{-1}} t_1 = {s^t}_{(gh^{-1})^t}$.
Thus, to prove~\eqref{eq:Peiffer-gen} it suffices to show the following simpler relation:
\begin{equation} \label{eq:Peiffer-simple} t^{-1}_1 s_g t_1 = {s^t}_{g^t} \text{ for all $s, t \in S_n$, $g\in G^n$.}\end{equation}

The key step in the proof is the following lemma.
\begin{lemma} \label{lem:transp-deff} 
 The relation~\eqref{eq:Peiffer-simple} holds in the special case when $s=(ij)$ and $t=(kl)$ are two nonequal transpositions. \end{lemma}
\begin{proof} 
First of all, we immediately check that~\eqref{eq:Peiffer-simple} holds in the special case when $t$ fixes $g'\in G^n$.
Indeed, by~\eqref{eq:main_rel} we have $t_1 = t_{g'}$, hence 
\begin{equation} \nonumber t^{-1}_1 s_{g'} t_1 = t^{-1}_{g'} s_{g'} t_{g'} = (s^t)_{g'} = (s^t)_{g'^t}.  \end{equation}

Without loss of generality we may assume that $l\neq i$ and $l\neq j$.
Denote by $g'$ the vector which differs from $g$ only at $l$-th position, for which we set $g'_l = g_k$. 
Since the only nontrivial component of $g'g^{-1}$ (resp. $g'g^{-t}$) is located at $l$-th (resp. $k$-th) position,
it is fixed by $s$ (resp. $s^t)$, hence from~\eqref{eq:main_rel} we conclude that $s_g = s_{g'}$ (resp. $(s^t)_{g'} = (s^t)_{g^t}$).
Finally, since $g'$ is fixed by $t$, we get that
\begin{equation} \nonumber t^{-1}_1 s_g t_1 = t^{-1}_1 s_{g'} t_1 = (s^t)_{g'^t} = s^t_{g'} = (s^t)_{g^t}. \text{\qedhere}\end{equation} \end{proof}

\begin{proof}[Proof of~\cref{thm:cms}]
Let us show that~\eqref{eq:Peiffer-simple} holds for arbitrary transpositions $s, t \in S_n$.
It suffices to consider the case $s=t=(ij)$. 
After choosing some $k\neq i,j$ and presenting $(ij)$ as $(kj)(ik)(kj)$ we use the lemma:
\begin{equation} \nonumber (ij)^{-1}_1 (ij)_g (ij)_1 = (ij)^{-1}_1 (kj)_g (ik)_g (kj)_g (ij)_1 =
(ki)_{g^{(ij)}} (jk)_{g^{(ij)}} (ki)_{g^{(ij)}} = (ij)_{g^{(ij)}}. \end{equation}

Proposition now follows by induction on the length of permutations $s$, $t$.
\end{proof}

\subsection{An explicit presentation of $S_n(G)$}
In this section we obtain an explicit presentation of $S_n(G)$ similar in appearance to the presentation of parametrized braid groups from~\cite{LS05}. % by means of generators and relations.
This presentation turns out to be much more economical than the original definition given in section~\ref{sec:QnG-def}.
An explicit presentation for the subgroup $HS_n(G)$ is derived from this presentation via Reidemeister-Schreier method.

We start with the following simple lemma which is, in essence, a variant of the standard presentation of a Weyl group in terms of reflections
 with some redundant generators and relations added (cf.~\cite[Theorem~2.4.3]{Ca89}).
\begin{lemma} \label{lm:Snpres} For $n\geq 3$ The symmetric group $S_n$ admits presentation with 
transpositions as the set of generators and the following list of defining relations
(in every formula distinct letters denote distinct indices):
\begin{align}
(ij)^2 = &\, 1,         \label{Sym1} \tag{S1} \\
(ij)^{(jk)} = &\, (ik), \label{Sym2} \tag{S2} \\
[(ij), (kl)] = &\,1.    \label{Sym3} \tag{S3} \\
(ij) =&\, (ji),         \label{Sym0} \tag{S4}
\end{align}
\end{lemma}

Now we are ready to formulate the main result of this subsection.
\begin{prop} \label{prop:Q-pres} For $n\geq 3$ and arbitrary group $G$ the group $S_n(G)$ admits presentation with the set of generators 
$\{(ij)_a \mid i\neq j, 1\leq i,j,\leq n, a\in G\}$ and the following list of relations (as before, distinct letters denote distinct indices):
\begin{align}
(ij)_a^2 = &\, 1,                \label{Q1} \tag{SG1} \\
(ij)_a^{(jk)_b} = &\, (ik)_{ab}, \label{Q2} \tag{SG2} \\
[(ij)_a, (kl)_b] = &\,1.         \label{Q3} \tag{SG3} \\
(ij)_a =&\, (ji)_{a^{-1}}        \label{Q4} \tag{SG4} 
\end{align}
\end{prop}
\begin{proof}
Denote by $S'$ the group from the statement of the proposition.
For $1\leq i\leq n$ and $x\in G$ denote by $x[i]$ the element of $G^n$ 
 whose only nontrivial component equals $x$ and is located in the $i$-th position.

It is not hard to deduce from the definition of $S_n(G)$ and~\cref{lem:transp-deff} 
 that the formula $(ij)_a \mapsto (ij)_{a[j]}$ gives a well-defined map $\varphi\colon S'\to S_n(G)$.

Now we are going to construct the map $\psi\colon S_n(G)\to S'$ in the opposite direction. 
Using the presentation of $S_n$ given by~\cref{lm:Snpres}  we define for a fixed $g\in G^n$ the map 
 $\psi_g\colon S_n \to S'$ by $\psi_g((ij)) = (ij)_{g_i^{-1} g_j}.$
It is obvious that $\psi_g$ preserves the defining relations \eqref{Sym1}--\eqref{Sym0} of $S_n$. 
It remains to show that the equation $\psi_g(s) = \psi_h(s)$ holds whenever $g, h\in G^n$ and $s\in S_n$ satisfy the requirement of~\eqref{eq:main_rel}.

Indeed, if $hg^{-1}$ is fixed by $s$ then for every $1\leq i\leq n$ we have $(hg^{-1})_i = (hg^{-1})_{s(i)}$, or equivalently
$h_i^{-1} h_{s(i)} = g_i^{-1} g_{s(i)}$. %This already shows that the identity $\psi_g(s) = \psi_h(s)$ holds in the special case when $s$ is a transposition. 
For example, if $s$ is a cycle of length $p$, i.\,e. $s=(i_1, i_2, \ldots i_p)$ with $i_{k+1} = s(i_k)$ we get that
\begin{multline} \nonumber
 \psi_g(s) = \psi_g\left({\prod\limits_{k=1}^{p-1}(i_k, i_{k+1})}\right) = 
 \prod\limits_{k=1}^{p-1}\left(i_k, i_{k+1}\right)_{g_{i_k}^{-1} g_{i_{k+1}}} = 
 \prod\limits_{k=1}^{p-1}\left(i_k, i_{k+1}\right)_{h_{i_k}^{-1} h_{i_{k+1}}} = \psi_h(s).
\end{multline}
The proof for a general $s$ is almost the same.
Verification of the fact that $\psi$ and $\varphi$ are mutually inverse is also immediate.
\end{proof}

We now obtain a presentation of the subgroup $HS_n(G)$.
For every $i\neq j$ and $a\in G$ we define the element $h_{ij}(a) \in HS_n(G)$ as follows:
\begin{equation} \label{eq:h-def} 
h_{ij}(a) = (ij)_{a} \cdot (ij)_1. 
\end{equation}
It is not hard to show that $h_{ij}(a)$ form a generating set for $HS_n(G)$.
In fact, there is an explicit formula how an element of $HS_n(G)$ originally expressed through $(ij)_{a}$'s can be rewritten in terms of $h_{ij}(a)$.
Indeed, if $h$ lies in $HS_n(G)$ and is written as $\prod_{k=1}^N(i_k j_k)_{a_k}$ for some $i_k\neq j_k$ and $a_k\in G$ then it can be rewritten as follows:
\begin{equation} \label{eq:rp} \tag{$\tau$}
 h = \prod_{k=1}^N h_{\sigma_k(i_k), \sigma_k(j_k)}(a_k),\text{ where } \sigma_k=\prod_{s=1}^{k-1} (i_s j_s) \in S_n. \end{equation}

We briefly recall the notion of a {\it rewriting process} given in~\cite[\S~2.3]{MKS76}.
If $G$ is a group presented by generators $a_\nu$ and relations $R_{\mu}(a_{\nu})$ and $H$ is its subgroup with a generating set $J_i(a_\nu)$ then
 a {\it rewriting process for $H$} is a function which maps every word $u$ in alphabet $a_\nu$ to a word $v$ in alphabet $s_i$ such that
 $u$ and $v[s_i:=J_i]$ define the same element of $G$ whenever $u$ represents an element of $H$.

With this terminology, the mapping~\eqref{eq:rp} defined above is a rewriting process for the subgroup $HS_n(G)$.
Since it does not arise as a rewriting process corresponding to a coset representative function,
this process is {\it not} a Reidemeister rewriting process in the sense of~\cite[\S~2.3]{MKS76}.
However, it still satisfies the following two key properties of a Reidemeister rewriting process (cf. with (v) and (vi) of~\cite[\S~2.3]{MKS76}): 
\begin{itemize}
 \item if $U$ and $U^*$ are freely equal words in $(ij)_a$ then $\tau(U)$ and $\tau(U^*)$ are also freely equal words in $h_{ij}(a)$;
 \item if $U_1$ and $U_2$ are two words in $(ij)_a$ which define elements of $HS_n(G)$ then the words $\tau(U_1U_2)$ and $\tau(U_1) \tau(U_2)$ are equal.
\end{itemize}
Using these two properties and repeating the arguments used in the proof of~\cite[Theorem~2.8]{MKS76}
one can simplify the generic presentation of $HS_n(G)$ given by~\cite[Theorem~2.6]{MKS76} and obtain the following.
\begin{lemma} \label{lm:h-gen}

 For $n\geq 3$ the group $HS_n(G)$ admits presentation on the generators $h_{ij}(a)$ with the following two families of defining relations:
 \begin{align}
  h_{ij}(a) = &\, \tau\left((ij)_a \cdot (ij)_1\right); \label{eq:tau1} \\
  \tau(KRK^{-1}) = &\, 1, \label{eq:tau2} 
 \end{align}  
 where $R$ varies over relations of~\cref{prop:Q-pres} and $K$ is any word in $(ij)_1$, $i\neq j$.
\end{lemma}

From~\eqref{eq:tau1} one immediately obtains the equality $h_{ij}(1)=1$, $i\neq j$.
Thus, if we denote by $\sigma$ the permutation corresponding to a word $K$ and let $S_n$ act on $h_{ij}(a)$'s in the natural way
 we will get that the word $\tau(KRK^{-1})$ is equivalent to ${}^{\sigma}\tau(R)$ (modulo relations $h_{ij}(1)=1$). 

Since the relations from~\cref{prop:Q-pres} are respected by the action of $S_n$,
 only relations of the form $\tau(R)=1$ are, in fact, needed for the presentation of $HS_n(G)$.
Writing down what $\tau(R)$ is for each of \eqref{Q1}--\eqref{Q4} we get the following.
 
\begin{prop} \label{prop:HSpres} For $n\geq 3$ the group $HS_n(G)$ admits presentation with generators $h_{ij}(a)$ and the following list of relations:
\begin{align}
h_{ij}(1)                     = &\, 1,              \tag{R0} \label{H0} \\
h_{ij}(a) h_{ji}(a)           = &\, 1,              \tag{R1} \label{H1} \\
h_{jk}(b) h_{ik}(a) h_{ij}(b) = &\, h_{ik}(ab),     \tag{R2} \label{H2} \\
[h_{ij}(a), h_{kl}(b)]        = &\, 1,              \tag{R3} \label{H3} \\
h_{ij}(a)^{-1}                = &\, h_{ij}(a^{-1}). \tag{R4} \label{H4}
\end{align}
\end{prop}

\section{Relationship with extensions of type \texorpdfstring{$\mathfrak{H}_n(G)$}{Hn(G)}} \label{sec:Hnextensions}
We start by briefly reviewing the material of \S~1--3 of~\cite{Reh78}. 
Let $n \geq 3$ be a natural number. By definition, the group $H_n(G)$ is given by generators
$h_{ij}(u)$, $u\in G$, $i\neq j$ and the following relations (cf. \cite[H1--H5~of~\S~2]{Reh78}):
\begin{align}
h_{ij}(u) h_{ji}(u)                = &\, 1,                        &                     \tag{H1} \label{RH1} \\
h_{ij}(u) h_{ki}(u) h_{jk}(u)      = &\, 1,                        &                     \tag{H2} \label{RH2} \\
h_{ij}(u) h_{ik}(v) h_{ij}(u)^{-1} = &\, h_{ik}(uv) h_{ik}(u)^{-1},& \text{for } j\neq k \tag{H3} \label{RH3} \\
h_{ij}(u) h_{kj}(v) h_{ij}(u)^{-1} = &\, h_{kj}(vu) h_{kj}(u)^{-1},& \text{for } i\neq k \tag{H4} \label{RH4} \\
[h_{ij}(u), h_{kl}(v)]             = &\, 1.                        &                     \tag{H5} \label{RH5} \end{align}
There is a surjective group homomorphism $H_n(G) \twoheadrightarrow D_n(G)$ sending $h_{ij}(g)$ to $d_{ij}(g)$.
Here $d_{ij}(g)$ stands for the element of $G^n$ whose
 $i$-th component equals $g$, $j$-th component equals $g^{-1}$ and all other components are trivial.

By definition, {\it an extension of type $\mathfrak{H}_n(G)$} is an extension $H$ of $D_n(G)$ that is also a quotient of $H_n(G)$, i.\,e. 
 the extension that fits into the following diagram.
\[ \xymatrix{ H_n(G) \ar@{->>}[rd] \ar@{->>}[d] & \\ H \ar@{->>}[r] & D_n(G)} \]
Any $\mathfrak{H}_n(G)$-extension is central, see~\cite[Proposition~2.3]{Reh78}.

We now recall the notion of \emph{an extension of type $\mathfrak{U}(G)$} introduced by Rehmann in~\cite[\S~1]{Reh78}.
Let $G$ be a group. Denote by $U(G)$ the group presented by generators $(u, v)$, $u,v\in G$ and relations:
\begin{align}
 ({}^{u} v, {}^{u}w) (u,  w) = &\, (uv, w), \tag{U1} \label{U1} \\
 (u,  vw) (v,  wu) (w,  uv)  = &\, 1.       \tag{U2} \label{U2}
\end{align}
There is a left action of $G$ on $U(G)$ given by ${}^x (u, v) = ({}^{x}u, {}^{x}v).$
Equivalently, $U(G)$ can be defined by the following relations, see~\cite[p.~8]{De76}.
\begin{align} 
 (u, v) (v, u)     = &\, 1,          \label{U4} \\
 (u, v) \ {}^v(u, w) = &\, (u, vw),  \label{U6} \\
 {}^u(v, w)(w, v)  = &\, (u, [v, w]).\label{U9} \end{align}
It is clear from the definition that $U(G)$ is a quotient of the nonabelian tensor square $G\otimes G$ and that
the map $(u, v) \mapsto [u, v]$ defines a $G$-equivariant map $U(G) \twoheadrightarrow [G, G]$.
Notice that in the literature one can find alternative notations for $U(G)$ which emphasize its
 connection with nonabelian tensor products e.\,g. $(G, G)$ or $G \mathbin{\widetilde{\wedge}} G$.
 
By definition, an {\it extension of type $\mathfrak{U}(G)$} is simply a quotient of $U(G)$
 that fits into the following commutative diagram with $G$-equivariant arrows.
\[ \xymatrix{ U(G) \ar@{->>}[rd] \ar@{->>}[d] & \\ U \ar@{->>}[r] & [G, G]} \]

There is a correspondence between extensions of type $\mathfrak{H}_n(G)$ and $\mathfrak{U}(G)$.
One can obtain a $\mathfrak{U}(G)$-extension from given $\mathfrak{H}_n(G)$-extension via restriction.
Indeed, let $H$ be an extension of type $\mathfrak{H}_n(G)$ and $k$ be arbitrary number $1\leq k\leq n$.
Denote by $\iota_k$ the inclusion map $[G, G] \to D_n(G)$ that inserts an element $g \in [G, G]$ at the $k$-th position.
Then the corresponding extension $U_H$ is simply the pull-back of $H$ with respect to $\iota_k$.

In other words, $U_H$ is simply the subgroup of $H$ generated by 
symbols $c_{kj}(u, v)=h_{kj}(u) \cdot h_{kj}(v) \cdot h_{kj}(vu)^{-1}$ 
 (here $j$ is any index not equal $k$, $c_{kj}(u,v)$ does not depend on $j$, see~\cite[Proposition~2.1]{Reh78}).
It can be shown that $U$ is an extension of type $\mathfrak{U}(G)$ with kernel~$\mathrm{Ker}(\pi_U)$ 
 isomorphic to $\mathrm{Ker}(\pi_H)$, see~\cite[Propositions~2.2--2.3]{Reh78}.
In~\cite[\S~3]{Reh78} Rehmann also describes the inverse construction which assign to every $\mathfrak{U}(G)$-extension
 $U\twoheadrightarrow [G, G]$ and a number $n\geq 3$ some extension $H_U$ of type $\mathfrak{H}_n(G)$ that fits into the pull-back diagram above.
% TODO: Is this correspondence one-to-one?
% TODO: Does $G\otimes G$ correspond to $H_n(G)$ via this construction (and vice-versa)?

\subsection{A simpler presentation of $H_n(G)$}
The aim of this subsection is to obtain a presentation of $H_n(G)$ that is simpler than the original presentation: 4 instead of 5 relations that are also shorter. 
Our main result is the following.
\begin{prop} \label{prop:simpler} 
For $n\geq 3$ relations \eqref{H0}--\eqref{H4} imply \eqref{RH1}--\eqref{RH5}.
Moreover, for $n\geq 4$ relations \eqref{H0}--\eqref{H3} are equivalent to \eqref{RH1}--\eqref{RH5}.
\end{prop}
For $u, v\in G$ we define the following two symbols:
\[c_{ij}(u,v)=h_{ij}(u)h_{ij}(v)h_{ij}(vu)^{-1}, c'_{ij}(u,v)=h_{ij}(u)h_{ij}(v)h_{ij}(uv)^{-1}.\]
Notice that~\eqref{RH4} implies $h_{ij}(1)=1$ therefore $c_{ij}(u, u^{-1}) = c'_{ij}(u, u^{-1})$.
Our goal is to show that~\eqref{RH3} can be omitted from the definition of $H_n(G)$ provided $n \geq 4$.
\begin{lemma} \label{item-lem33} If one excludes relation~\eqref{RH3} from the list of defining relations for the group $H_n(G)$, $n\geq 3$,
 the following still remain true:
 \begin{lemlist}
\item \label{item-lem33-cntr} The elements $c_{ij}(u, u^{-1})$ are central in $H_n(G)$;
\item \label{item-lem33-comm} One has $c'_{ij}(u, v) = [h_{ij}(u), h_{kj}(v)]$, $k\neq i, j$;
\item \label{item-lem33-conj}  One has ${}^{h_{ij}(w)}c_{kj}'(u, v) = c_{kj}'(u, w)^{-1} c_{kj}'(u, vw)$, $k\neq i, j$;
\item \label{item-lem33-conj2} One has ${}^{h_{ij}(w)} c'_{ij}(u, v) = c'_{ij}(uw, v) c'_{ij}(w, v)^{-1}$.
 \end{lemlist}
\end{lemma}
\begin{proof}
First one shows using~\eqref{RH4} that $c_{ij}(u, u^{-1})$ centralizes $h_{kj}(v)$ (cf. with the proof of~\cite[Lemma~2.1(2)]{Reh78}).
Since in any group $[a, b]=1$ implies $[a^{-1}, b] = [a, b^{-1}] = [a^{-1}, b^{-1}] = 1$ we get that $c_{ij}(u, u^{-1})$ also centralizes $h_{kj}(v)^{-1} = h_{jk}(v)$ and
 $c_{ij}(u^{-1}, u)^{-1} = c_{ji}(u, u^{-1})$ centralizes both $h_{kj}(v)$ and $h_{jk}(v)$.
Together with~\eqref{RH2} and~\eqref{RH5} this implies that $c_{ij}(u, u^{-1})$ centralizes all the generators of $H_n(G)$ and hence lies in the center of $H_n(G)$.

The second and third assertions are straightforward corollaries of~\eqref{RH4}.
The fourth assertion follows from the third one:
\[{}^{h_{ij}(w)} c'_{ij}(u, v) = {}^{h_{ij}(w)} c'_{kj}(v, u)^{-1} = (c'_{kj}(v, w)^{-1} c'_{kj}(v, uw))^{-1} = c'_{ij}(uw, v) c'_{ij}(w, v)^{-1}.\]
\end{proof}

\begin{lemma}
If one excludes relation~\eqref{RH3} from the list of defining relations for the group $H_n(G)$, $n\geq 3$, the following statements are equivalent:
\begin{enumerate}
\item \label{item1} \eqref{RH3} holds;
\item \label{item4} one has $c'_{ij}(u, v)^{-1} = c'_{ij}(v, u)$;
\item \label{item5} symbols $c'_{ij}(u, v)$ do not depend on $i$;
\item \label{item5} one has $c'_{ij}(u, vw) = c'_{ij}(wu, v) \cdot c'_{ij}(uv, w)$;
\item \label{item6} one has $c'_{ij}(u, vu^{-1}) = c'_{ij}(uv, u^{-1})$.
\end{enumerate}
\end{lemma}
\begin{proof}
Implications $(1) \implies (2) \implies (3) \implies (4)$ are essentially contained in the proof of~\cite[Lemmas~2.1-2.2]{Reh78}.
Implication $(4) \implies (5)$ is trivial.

We now prove $(5) \implies (1)$. Notice that~\eqref{RH4} implies ${}^{h_{ij}(u)^{-1}}h_{ik}(v) = h_{ik}(u)^{-1} h_{ik}(vu)$, therefore
using~\cref{item-lem33-cntr} we get that ${}^{h_{ij}(u)}h_{ik}(v) = {}^{c_{ij}(u, u^{-1}) h_{ij}(u^{-1})^{-1}}h_{ik}(v) = h_{ik}(u^{-1})^{-1} h_{ik}(vu^{-1}).$
Thus, \eqref{RH3} is equivalent to the equality 
\[ h_{ik}(uv) h_{ik}(u)^{-1} = h_{ik}(u^{-1})^{-1} h_{ik}(vu^{-1}),\]
or what is the same
\[c'_{ik}(u^{-1}, uv) = h_{ik}(u^{-1}) h_{ik}(uv) =  h_{ik}(vu^{-1}) h_{ik}(u) = c'_{ik}(vu^{-1}, u).\]
It is clear that the last equality is an equivalent reformulation of $(5)$.
\end{proof}

\begin{proof}[Proof of~\cref{prop:simpler}]
The assertion of the proposition becomes evident once one notices the following:
\begin{itemize}
 \item \eqref{RH2} and \eqref{RH4} are implied by \eqref{H0}--\eqref{H2};
 \item \eqref{H0} follows from \eqref{RH4};
 \item \eqref{H2} follows from \eqref{RH1}, \eqref{RH2} and \eqref{RH4};
 \item \eqref{RH4} and \eqref{RH3} are equivalent in the presence of \eqref{H4};
 \item for $n\geq 4$ the third statement of the previous lemma follows from~\cref{item-lem33-comm}.
\end{itemize}
\end{proof}

\begin{cor} \label{cor:main} For $n \geq 3$ the group $HS_n(G)$ is the quotient of the extension $H_n(G)$ by~\eqref{H4} and, in particular, is an $\mathfrak{H}_n(G)$-extension.
The associated $\mathfrak{U}(G)$ extension $U_{HS_n(G)}$ is isomophic to the quotient of $U(G)$ modulo single family of relations
\begin{equation} \nonumber (u, u) = 1, \ u\in G. \end{equation}
In particular, we have the following natural isomorphism:
\begin{equation} \Ker(HS_n(G) \to D_n(G)) = \Ker(S_n(G) \to G \wr S_n) \cong \HH_2(G, \ZZ). \end{equation} \end{cor}
\begin{proof}
Most of the assertions follow from~\cref{prop:simpler} and~\cite[Proposition~5]{De76}.
One also has to use the formula $c_{ij}(u, u) = c_{ij}(u, u^{-1})$ (see~\cite[p.~87]{Reh78}).
\end{proof}

\section{Topological applications} \label{sec:main}
In this section we prove topological claims made in the introduction.
\subsection{Proof of main results}
Recall that for arbitrary set $X$ one defines the space $EX$ as the the simplicial set whose set of $k$-simplices $EX_k$
 is $X^{k+1}$ and whose faces and degeneracies are obtained by omitting and repeating components. 
For a group $G$ we denote by $\pi_G$ the canonical map $EG \to BG$ sending $(g, h) \in EG_1$ to $g^{-1}h \in BG_1$.
 
Now let $N$ be a group acting on $X$. We define two simplicial sets $U$ and $V$ as follows:
\[ U = \bigcup\limits_{n\in N} E(\Gamma_{n.-}) \subseteq E(X\times X),\ \ V = \bigcup\limits_{x,y\in X}E(N(x\to y)) \subseteq EN. \]
Here $\Gamma_{n.-}$ is the graph of the function $(x \mapsto nx)$ and $N(x\to y)$ denotes the subset of elements $n\in N$ satisfying $nx=y$.
With this notation the subset $N(x\to x)$ coincides with the stabilizer subgroup $N_x \leq N$.

\begin{lemma} \label{lm:quillen-a} The simplicial sets $U$ and $V$ are homotopy equivalent. \end{lemma}
\begin{proof} First, we define yet another simplicial set $W$ as follows.
Its $k$-simplices $W_k$ are matrices $\left(\begin{smallmatrix}x_0 & x_1 & \ldots & x_k&\\ n_0 & n_1 & \ldots & n_k \end{smallmatrix}\right)$,
 where $x_i\in X$ are $n_i\in N$ are such that all $n_i$'s act each $x_j$ in the same way, i.\,e. $n_ix_j = n_{i'} x_j$ for $0\leq i,i',j\leq k$. 
 The faces and degeneracies of $W$ are the maps of omission and repetition of columns.
 
 Now there are two simplicial maps $f\colon W\to U$, $g\colon W\to V$ whose action on $0$-simplicies is given by 
  $f\left(\begin{smallmatrix}x_0 \\ n_0\end{smallmatrix}\right) = (x_0, n_0x_0)$, 
  $g\left(\begin{smallmatrix}x_0 \\ n_0\end{smallmatrix}\right) = n_0$. 
 To prove the lemma it suffices to show that $f$ and $g$ are homotopy equivalences. 
 The proof for $f$ and $g$ is similar, let us show, for example, that $g$ is a homotopy equivalence.
 
 In view of Quillen theorem A (cf.\cite[ex.~IV.3.11]{Kbook}) it suffices to show that for each $p$-simplex $d \colon \Delta^p \to V$ the 
  pullback $g/(p, d)$ of $d$ and $g$ is contractible.
 The simplicial set $g/(p, d)$ can be interpreted as the subset of $\Delta^p \times E(X)$ whose set of $k$-simplices consists of pairs
  $(\alpha\colon \underline{k}\to \underline{p}, (x_0, \ldots, x_k)\in X(\alpha, d)^{k+1})$.
 Here $X(\alpha, d)$ is the subset of $X$ consisting of all $x$ for which $d_{\alpha(i)}x = d_{\alpha(j)}x$ for $0\leq i,j\leq k$.
 Notice that the set $X_d := X(id_{\underline{p}}, d)$ is nonempty and is contained in every $X(\alpha, d)$ (it even equals $X(\alpha, d)$ for surjective $\alpha$).
 Now choose a point $\widetilde{x}\in X_d$ and consider the simplicial homotopy \[H\colon \Delta^p \times EX \times \Delta^1 \to \Delta^p\times EX\] 
  between the identity map of $\Delta^p \times EX$ and
 the map $\Delta^p \times c_{\widetilde{x}}$, where $c_{\widetilde{x}}$ is the constant map. 
 More concretely, $H$ sends each triple $(\alpha\colon \underline{k} \to \underline{p}, (x_0, \ldots, x_k), \beta\colon \underline{k}\to\underline{1})$
 to $(\alpha, (x_0, \ldots, x_{i-1}, \widetilde{x}, \ldots, \widetilde{x}))$, where $i$ is the minimal number such that $\beta(i)=1$.
 By the choice of $\widetilde{x}$ the image of $H$ restricted to $g/(p, d)\times \Delta^1$ is contained in $g/(p, d)$, hence $g/(p, d)$ is contractible. 
\end{proof}

Now suppose that $X=H$ is also a group upon which $N$ acts on the left.
\begin{cor} \label{cor:ker-iso}
Consider the following two simplicial sets:
\[ S = \bigcup\limits_{h\in H} BN_h \subseteq BN,\ \ \ T = \pi_N(U) \subseteq B(H \times H).\]
There is an isomorphism $\Ker(\pi_1(S) \to N) \cong \Ker(\pi_1(T) \to H \times H).$
Moreover, the higher homotopy groups of $S$ and $T$ are isomorphic. \end{cor}
\begin{proof}
Consider the following two pull-back squares:
\[ \xymatrix{ V  \ar@{^{(}->}[r] \ar[d] \pullbackcorner & \ar[d]^{\pi_N} EN \\
              S \ar@{^{(}->}[r] & BN } \ \ \ 
   \xymatrix{ U  \ar@{^{(}->}[r] \ar[d] \pullbackcorner & \ar[d]^{\pi_{H \times H}} E(H \times H) \\
              T \ar@{^{(}->}[r] & B(H \times H)}\] 
The required isomorphism can be obtained from the homotopy long exact sequence applied to the left arrows of these diagrams.\end{proof}

Now let $G$ be a group. Set $N = G \wr S_n$, $H = G^n$ and consider the left action of $N$ on $H$ given by $(g, s) \cdot h = gh^{s^{-1}}$, $g, h\in G^n$, $s\in S_n$.
If one reorders the components of $BG^{2n}$ accordingly, the simplicial subset $T \subset BG^{2n}$ from the above corollary
 becomes precisely the preimage of $0$ under $h_n \colon BG^{2n} \to \ZZ[BG]$.

It is also easy to compute the map $\pi_1(S) \to N$. Indeed, van Kampen theorem~\cite[Theorem~2.7]{May99} asserts that
$\pi_1(S)$ is isomorphic to the free product of stabilizer subgroups $N_{h} \leq N$ amalgamated over pairwise intersections $N_h \cap N_{h'}$, $h, h\in H$.
For $h \in G^n$ the subgroup $N_h$ consists of elements $(g, s) \in N$ satisfying $gh^{s^{-1}} = h$, i.e. elements of the form $(hh^{-s^{-1}}, s)$.
Thus, $N_h\cong S_n$, $\pi_1(S)$ is isomorphic to the group $S_n(G)$ and the map $\pi_1(S) \to N$ coincides with the map $\mu$ defined in section~\ref{sec:QnG-def}.

\begin{example} \label{ex44}
 We leave it as an exercise to the reader to check that an element $[h_{12}(x), h_{13}(y)] \in \pi_1(S)$ corresponds to the following element of $\pi_1(T)$
  under the isomorphism of \cref{lm:quillen-a}:
 \begin{multline} \label{eq:pathT}
  \gamma_{x,y} = (x[1], x[1]) \circ (x^{-1}[1], x^{-1}[2]) \circ (y[1], y[1]) \circ (y^{-1}[1], y^{-1}[3]) \circ \\ \circ (x[1], x[2]) \circ (x^{-1}[1], x^{-1}[1]) \circ (y[1], y[3]) \circ (y^{-1}[1], y^{-1}[1])).
 \end{multline}  
\end{example}

\begin{proof}[Proof of~\cref{thm:main}]
Factor $h_n$ as a composition of a trivial cofibration followed by a fibration: 
\begin{equation} \label{eq:fibr-repl} \xymatrix{BG^{2n} \ar@{^{(}->}[r] & E_{h_n} \ar@{->>}^(.45){ev_1 \circ \pi_2}[r] & \ZZ[BG]} \end{equation}
For example, we can define $E_{h_n}$ and the homotopy fiber $F_{h_n}$ via the usual path space construction
 (here we use the fact that $BG$ and $\ZZ[BG]$ are fibrant).
\[ \xymatrix{ E_{h_n}  \ar[r]_{\pi_2} \ar[d]_{\pi_1} \pullbackcorner & \ar[d]^{ev_0} \ZZ[BG]^I\\
              BG^{2n} \ar[r]_{h_n} & \ZZ[BG] } \ \ \ 
   \xymatrix{ F_{h_n}  \ar[r] \ar[d] \pullbackcorner & \ar[d]^{ev_1\,\circ\,\pi_2} E_{h_n} \\
              pt \ar[r]_{0} & \ZZ[BG]}  \]
Now write down the starting portion of the long homotopy exact sequence of the fibration $E_{h_n} \twoheadrightarrow \ZZ[BG]$ and
 denote by $K$ the kernel of the map $\nu\colon \pi_1(T) \to G^{2n}$ induced by the embedding $h_n^{-1}(0)=T \subseteq BG^{2n}$.
We come to the following commutative diagram:
\[ \xymatrix{1 \ar[r] & K       \ar[d]_{\psi} \ar[r] & \pi_1(T) \ar[d]_{\phi} \ar[r]_{\nu}     & G^{2n} \ar[r] \ar[d]_{\cong} & \HH_1(G, \ZZ) \ar[r] \ar@{=}[d] & 1 \\
             1 \ar[r] & \HH_2(G, \ZZ) \ar[r] & \pi_1(F_{h_n})           \ar[r]  & \pi_1(E_{h_n}) \ar[r]                       & \HH_1(G, \ZZ) \ar[r] & 1}\]
We already know by Corollaries~\ref{cor:main} and~\ref{cor:ker-iso} that $K$ is naturally isomorphic to $\HH_2(G, \ZZ)$ provided $n\geq 3$.

Let us show that $\psi$ is an isomorphism. Assume for a moment that $G$ is an abelian group.
In this case a generator $x \wedge y$ of $\HH_2(G, \ZZ)$ in Miller's presentation corresponds to the class of 2-cycle $c = (x, y) - (y, x)$ (cf.~\cite[(14), p.~582]{Mi52}), which,
  in turn, corresponds to the following normalized 2-cycle:
\begin{equation} \label{eq:normalized} c' = c - s_0d_0c - s_1d_1c + s_1d_0 = (x, y) - (y, x) - (x, 0) +(y, 0) + (0, x) - (0, y). \end{equation}
 
Since $F_{h_n}$ and $\ZZ[BG]$ are fibrant, any element of $\pi_1(F_{h_n}, 0)$ can be represented with 
 some 1-simplex $(\gamma, s) \in G^{2n} \times \ZZ[BG]_2$ satisfying $d_2(s) = h_n(\gamma)$, $d_0(s) = d_1(s)=0$. 
It remains to find in $F_{h_n}$ the 1-simplex homotopic to the path $\phi(\gamma_{x,y})$ (cf.~\cref{ex44}).

We use the following inductive procedure. Let $\gamma_0$ be an initial segment of a path $\gamma$ in $T$ for which we have already found such simplex
 $(g^0, s^0)$ in $F_{h_n}$ and let $g \in T_1$ be the next chain link. 
Set $s_0 = \sum (a_j, b_j) \in\ZZ[BG]_2$, $t = h_n (g^0, g) = \sum_{i=1}^{2n}(-1)^i(g^0_i, g_i)$
and compute the filler for the $3$-horn $(0,\ , s_0, t)$. Denote the $1$-st face of this filler by $s_1$.
It is clear that $(g^0g, s_1)$ is the image of $\gamma^0 \circ g$ via $\phi$.
The concrete formula for $s_1$ can be chosen e.\,g. as follows:
\[s_1 = \sum_{i=1}^{2n}(g^0_i + g_i, -g_i) - \sum_j (a_j + b_j, -b_j).\]

Applying the above recipe to the path $\gamma_{x, y}$ we get an expression equal to~\eqref{eq:normalized}.
 This shows that $\psi$ is the identity map for an abelian group $G$.
Since $\psi$ is natural in $G$, by~\cref{lm:endotr} below we get that $\psi$ is an isomorphism for arbitrary $G$,
 therefore $\phi$ is also an isomorphism, as claimed.
\end{proof}

\begin{lemma} \label{lm:endotr} The only natural endotransformations of the second homology functor $\HH_2(-, \ZZ)\colon \catname{Groups}\to \catname{Ab}$ 
 are morphisms of multiplication by $n \in \ZZ$.
\end{lemma}
\begin{proof}
 Denote by $\eta$ an endotransformation $\HH_2(-, \ZZ) \to \HH_2(-, \ZZ)$.
 When restricted to the subcategory of free finitely-generated abelian groups $\catname{Add}(\ZZ) \subseteq \catname{Ab}$ the second homology functor
 coincides with the second exterior power functor $A \mapsto \wedge^2A$. 
 
 Recall from~\cite[Theorem~6.13.12]{Ba96} that the category of quadratic functors is equivalent to the category of quadratic $\ZZ$-modules (see Definition~6.13.5 ibid.)
 The functor $A \mapsto \wedge^2A$ is clearly quadratic and corresponds to the quadratic $\ZZ$-module $0 \to \ZZ \to 0$ under this equivalence.
 Thus, we get that $\eta$ restricted to $\catname{Add}(\ZZ)$ coincides with the the morphism of multiplication by $n\in \ZZ$.
 
 Consider the group $\Gamma_k = \langle x_1, y_1, \ldots x_k, y_k \mid [x_1, y_1]\cdot \ldots \cdot [x_k, y_k] \rangle$ (the fundamental group of a sphere with $k$ handles).
 It is clear that the abelianization map $\Gamma_k \to \ZZ^{2k}$ induces an injective map $\HH_2(\Gamma_k, \ZZ) \cong \ZZ \to \wedge^2\ZZ^{2k}$.  
 Consider the following diagrams.
  \[ \xymatrix{ \ZZ \ar@{^{(}->}[r] \ar[d]_{\eta_{\Gamma_k}} & \ar[d]^{n \cdot} \wedge^2\ZZ^{2k} \\
                \ZZ \ar@{^{(}->}[r]                          & \wedge^2\ZZ^{2k} } \ \ \ 
     \xymatrix{   \ZZ  \ar[r]^(.35){\chi} \ar[d]_{\eta_{\Gamma_k}} & \ar[d]^{\eta_G} \HH_2(G, \ZZ) \\
                  \ZZ  \ar[r]^(.35){\chi}                          & \HH_2(G, \ZZ)}  \]
 From the left diagram it follows that $\eta_{\Gamma_k}$ is also the morphism of multiplication by $n$.
 For every element $x\in \HH_2(G, \ZZ)$ there exist an integer $k$ and a map $\chi\colon \Gamma_k\to G$ sending the generator of $\HH_2(\Gamma_k, \ZZ)$ to $x$.
 From the right square we conclude that $\eta_G(x) = nx$, as claimed.
\end{proof}

\subsection{Concluding remarks}
We finish the paper by showing that the map $h_\infty$ is a quasifibration.
In order to do this we use a simplicial version of the so-called ''delayed homotopy lifting property``, cf. {\bf ???}.

\begin{df} \label{df:dhlp}
 Let $C$ be a class of injective maps in the category of simplicial sets.
 We say that a map of simplicial sets $p\colon E \to B$ satisfies (simplicial) {\it delayed homotopy lifting property} with respect to $C$ if 
  for every commutative square
\begin{equation} \nonumber \xymatrix{ 
U \ar[rr]^{g}  \ar@{^{(}->}_{i}[d] & & E \ar^{p}[d] \\
V \ar[rr]_{F} \ar@{-->}[rru]^{\widetilde{F}} & & B}
\label{eq:plp} \end{equation}
in which $i \in C$ there exists a map $\widetilde{F}$ so that the lower triangle commutes strictly and the upper one commutes up to homotopy $H$
 such that $pH \colon U \times I \to B$ coincides with the composite $U \times I \xrightarrow{\pi} U \xrightarrow{g} B $.
\end{df}

Denote by $C_{pr}$ the class of retractible inclusions between compact polyhedral simplicial sets.
In other words, for polyhedral $U$ and $V$ an inclusion $i\colon U\to V$ lies in $C_{pr}$ if there is a map $r\colon V\to U$ such that $ri=id_U$.
\begin{lemma} \label{lm:dhlp} Let $X$ be a fibrant connected simplicial set.
The map $h \colon X^\infty \to \ZZ[X]$ satisfies the delayed homotopy lifting property with respect to the class $C_{pr}$.
\end{lemma}
\begin{proof}
 Notice first that the original question can be reduced to the special case $g=0$.
 Indeed, since $U$ is finite the image of $g$ is contained in $X^{2N} \subset X^\infty$. Set $F_0 = F - hgr$.
 If we could find a lift $\widetilde{F}_0$ for a pair of maps $(g=0, F_0)$ then $\widetilde{F} = gr \oplus \widetilde{F}_0[N]$
 would be a solution to the original lifting problem for $(g, F)$.
 
 Now choose a contractible fibrant simplicial set $W$ which maps surjectively onto $X$ (e.\,g. set $W=\underline{\Hom}(\Delta[1], X)$ ???).
 Since $p \colon W \twoheadrightarrow X$ is surjective, the associated map between free simplicial abelian groups is a (Kan) fibration
  and we can choose a lift $\widetilde{F}$ in the following diagram.
\begin{equation} \nonumber \xymatrix{ 
U \ar[r]^{0}  \ar@{^{(}->}_{i_0}[d] & \ZZ[W] \ar@{->>}[d]  \\
V \ar[r]_{F} \ar[ru]_{\widetilde{F}} & \ZZ[X]}
\end{equation}
 Notice that if the lifting problem for $(0, \widetilde{F})$ and $h_W \colon W^\infty \to \ZZ[W]$ has a solution, 
  then so does the original problem for $(0, F)$ and $h_\infty$.

 By~\cite[Lemma~9.1]{Po17} we can lift $\widetilde{F}$ along $h_W$ so that the bottom triangle in the diagram
\begin{equation} \nonumber \xymatrix{ 
U \ar[r]^{0}  \ar@{^{(}->}_{i_0}[d]                               & W^\infty \ar[d]^{h_W}  \\
V \ar[r]_{\widetilde{F}} \ar@{-->}[ru]^{G} & \ZZ[W]}
\end{equation}
 commutes strictly. It is clear that the image of $Gi$ sits inside the fiber $F_{h_W, 0}$.
 It remains to see that $F_{h_W, 0}$ is contractible (the contracting homotopy for $W^\infty$ can be restricted to $F_{h_W, 0}$).
 Thus, we get that $H \colon Gi \cong 0$ and, moreover, that $h_W H = 0$. \end{proof}

\begin{lemma} \label{lem:topo-facts} Let $p\colon (E, e) \to (B, b)$ be a map of based topological spaces.
Let $p^{-1}(b) \hookrightarrow F_{p}(b) \subseteq E\times_B B^I$ be the inclusion map of the fiber of $p$ into the homotopy fiber.
Denote $k$-th relative homotopy group $\pi_k(F_p(b), p^{-1}(b), e))$ by $G_k$.
\begin{lemlist}
 \item Maps of triples $(D^k, S^k, pt) \to (F_p(b), p^{-1}(b), e)$ are in one-to-one correspondence with commuting diagrams of the following form.
   \begin{equation} \xymatrix{ 
    D^k \ar@{^{(}->}_{i_0}[d]  \ar_{a}[rr] & & E \ar^{p}[d]  \\
    D^k\times I \ar^{\pi}[r] & (D^k \times I) / J \ar^(.7){A}[r] & B}
    \label{eq:plp} \end{equation}
    Here $J$ denotes $(S^k \times I) \cup (D^k \times \{1\})$. In the sequel we denote such diagrams by $(a, A)$.
 \item Two diagrams $(a_0, A_0)$ and $(a_1, A_1)$ represent the same element of $G_k$ iff there exists a family of maps $(a_t, A_t)$
     continously depending on $t\in [0, 1]$ so that $pa_t = A_t i_0$ holds for all $t$ (we call such family a homotopy of diagrams).
 \item Assume $A\pi$ lifts with respect to $p$ (i.\,e. there is $\widetilde{A}$ such that $\widetilde{A} \iota_0 = a$ and $p\widetilde{A}=A$), then the element of $G_k$ represented by $(a, A)$ is trivial.
 \item \label{item:continue} Let $(a, A)$ be a diagram and and $H$ be a homotopy between $a$ and any map $a'\colon D^k \to E$ such that $pH$ maps $S^k\times I$ to the point $b$.
       Then there exists a map $A'$ such that $(a', A')$ is diagram homotopic to $(a, A)$.
 \item \label{item:weaker} The claim of the third assertion remains true if instead of the strict equality $\widetilde{A} i_0 = a$ 
       we require that $\widetilde{A}i_0$ and $a$ are homotopic via some homotopy $H$ satisfying $pH(S^k\times I) = \{b\}$.
\end{lemlist}
\end{lemma}


\begin{lemma} \label{lm:weak-equiv} If a map $p\colon E \to B$ of simplicial sets satisfies simplicial delayed homotopy lifting property with respect to $C_{pr}$ 
 then its geometric realization is a quasifibration, i.\,e. for every point $b \in B_0$ the inclusion $|p|^{-1}(b) \hookrightarrow F_{|p|}(b)$ is a weak equivalence. \end{lemma}
\begin{proof}

 We want to prove that $\pi_k(F_{|p|}(b), |p|^{-1}(b))$ are all trivial for $k \geq 1$.
 Clearly, it suffices to show that for every diagram $(a, A)$ of the form~\eqref{eq:plp} the map $|p|$ 
  has a lift $\widetilde{A}$ satisfying the assumptions of~\cref{item:weaker}.
 
 We are going to approxime $(a, A)$ with a simplicial map and then invoke the lifting property of \ref{df:dhlp}.
 This is accomplished in a series of steps. %by means of the simplicial approximation theorem which we have to apply three times.
 \begin{itemize}
  \item Denote the restriction of $a$ to $S^k$ by $a_0$.
   Notice that the image of $a_0$ is contained in $|p^{-1}(b)|$.
   Using the simplicial approximation theorem~\cite[Theorem~4.7]{Jar04} (without initial conditions)
    we find a simplicial map $a_0'\colon \sd^m (S^k) \to p^{-1}(b)$ such that $a|\gamma^m| \cong |a'_0|$
     (here $\sd$ is the subdivision functor and $\gamma^m$ denotes the canonical natural transformation $\sd^m X \to X$)
  \item It is clear that there exists a map $a'\colon D^k\to |E|$ which continues $|a'_0|$ and is homotopic to $a$ via some homotopy $H$ satisfying $pH(S^k\times I) = \{b\}$.
  \item Invoke the approximation theorem once again (with the initial condition specified by $|a'_0|$) and find $q>m$ and $a''\colon \sd^{q}(D_k) \to E$
    such that $a'|\gamma^{q}| \cong a'$ rel $S^k$.
  \item Applying \cref{item:continue} we can continue the homotopy $a \cong a' \cong a''$ to a homotopy of diagrams $(a, A) \cong (a'', A')$.
  \item Applying the approximation theorem to $A'$ (with the initial condition on the boundary $\partial (D_k \times I) = D^k\times \{0\} \cup J$ given by 
    $a''$ and $const_b$, respectively) we find an integer $r>q$ and an approximation $A''\colon \sd^{r}(D_k \times I) \to B$ homotopic to $A'$ rel $\partial (D_k \times I)$
     so that $(a'', A'')$ is a commutative diagram of the following form
   \begin{equation} \nonumber \xymatrix{ 
    \sd^{r}(D^k)          \ar[r]^(0.7){a''\gamma^{r-q}}  \ar@{^{(}->}_{\sd^{r}(i)}[d] & E \ar[d]^{p}  \\
    \sd^{r}(D^k \times I) \ar[r]_(0.7){A''}  \ar@{-->}[ru]_{\widetilde{A}} & B}
   \end{equation}
 \end{itemize}

 Since $\sd(-)$ is a functor the map $\sd^{r}(i)$ is retractible and
  by the assumption there exists a lift $\widetilde{A}$ satisfying the assumptions of~\cref{item:weaker}.
 \end{proof}

\printbibliography

\end{document}